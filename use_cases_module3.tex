\begin{usecase}{Use Case name}
	% Opis przypadku użycia
	\desc{Use Case description}
	% Priorytet
	\priority{3}
	% Aktorzy
	\actors{Aktor1, Aktor2, Aktor3}
	% Przebieg główny
	\main{Main flow description}
	\begin{enumerate}
		\item Action 1
		\item Action 2
		\item Action 3
	\end{enumerate}
	% Przebiegi alternatywne
	\beginalternatives
	\alternative{Alternative flow 1}{Actions \ldots}
	\alternative{Alternative flow 2}{Actions \ldots}
	\alternative{Alternative flow 3}{Actions \ldots}
	% Diagram
	\diagram{todo.pdf}{0.6} % drugi argument to szerokość obrazu (1.0 = szerokość strony)
	% Diagram sekwencji
	\sekwencja{todo.pdf}{0.6}  % drugi argument to szerokość obrazu (1.0 = szerokość strony)
    % Diagram sekwencji
	\klasa{todo.pdf}{0.6}  % drugi argument to szerokość obrazu (1.0 = szerokość strony)
	% Pre-conditions
	\precond{\todo}
	% Post-conditions
	\postcond{\todo}
	% Referencje
	\references{\rf{1} \rf{2} \rnf{1} \rnf{2}}
\end{usecase}
% może nie usuwajcie tego szablonu z góry, będzie można sobie kopiować


\begin{usecase}{Przeglądanie Projektów}
\label{przegladanie_projektow}
	% Opis przypadku użycia
	\desc{Użytkownik przegląda Projekty, w których funkcjonuje w roli Członka Projektu}
	% Priorytet
	\priority{10}
	% Aktorzy
	\actors{Użytkownik, Członek Projektu}
	% Przebieg główny
	\begin{enumerate}
      \item Użytkownik wchodzi w sekcję "moje projekty"
      \item Użytkownik zaznacza opcję filtrowania projektów, w których jest członkiem
      \item System wyświetla listę projektów, w których Użytkownik funkcjonuje w roli Członka Projektu.
	\end{enumerate}
	% Diagram
	\diagram{przegladanie_projektow.pdf}{0.9} % drugi argument to szerokość obrazu (1.0 = szerokość strony)
	% Pre-conditions
	\precond{Użytkownik jest zalogowany (\ref{logowanie})}
	\references{\rf{3}}
\end{usecase}

\begin{usecase}{Tworzenie Projektu}
\label{tworzenie_projektu}
	% Opis przypadku użycia
	\desc{Użytkownik tworzy nowy Projekt}
	% Priorytet
	\priority{10}
	% Aktorzy
	\actors{Użytkownik}
	% Przebieg główny
	\begin{enumerate}
    \item Użytkownik przegląda istniejące Projekty (\ref{przegladanie_projektow})
    \item Użytkownik wybiera opcję  “Utwórz nowy projekt”
    \item Użytkownik wypełnia formularz
    \item Użytkownik zatwierdza utworzenie nowego projektu.
	\end{enumerate}
	% Diagram
	\diagram{tworzenie_projektu.pdf}{0.9} % drugi argument to szerokość obrazu (1.0 = szerokość strony)
	% Pre-conditions
	\precond{Użytkownik jest zalogowany}
	% Post-conditions
	\postcond{Został utworzony nowy Projekt, a Użytkownik, który go stworzył, zostaje Twórcą tego Projektu}
	\references{\rf{1}}
\end{usecase}


\begin{usecase}{Edycja Projektu}
	% Opis przypadku użycia
	\desc{Twórca Projektu edytuje swój Projekt}
	% Priorytet
	\priority{10}
	% Aktorzy
	\actors{Twórca Projektu}
	% Przebieg główny
	\begin{enumerate}
    \item Twórca Projektu przegląda istniejące Projekty (\ref{przegladanie_projektow})
    \item Twórca Projektu wybiera Projekt
    \item Twórca Projektu dokonuje zmian w formularzu
    \item Twórca Projektu zatwierdza zmiany
	\end{enumerate}
	% Diagram
	\diagram{edycja_projektu.pdf}{0.9} % drugi argument to szerokość obrazu (1.0 = szerokość strony)
	% Pre-conditions
	\precond{Użytkownik jest zalogowany (\ref{logowanie}) i jest Twórcą edytowanego Projektu (\ref{tworzenie_projektu})}
	% Post-conditions
	\postcond{Właściwości Projektu zostały zmienione}
	\references{\rf{9}}
\end{usecase}


\begin{usecase}{Zamknięcie projektu}
	% Opis przypadku użycia
	\desc{Twórca Projektu zgłasza do zamknięcia swój Projekt}
	% Priorytet
	\priority{10}
	% Aktorzy
	\actors{Twórca Projektu}
	% Przebieg główny
	\begin{enumerate}
	\item Twórca Projektu przegląda istniejące Projekty (\ref{przegladanie_projektow})
    \item Twórca Projektu wybiera Projekt
    \item Twórca Projektu wybiera Zgłoszenie
    \item Twórca Projektu wydaje polecenie zamknięcia projektu
	\end{enumerate}
	% Diagram
	\diagram{zamkniecie_projektu.pdf}{0.9} % drugi argument to szerokość obrazu (1.0 = szerokość strony)
	% Pre-conditions
	\precond{Użytkownik jest zalogowany (\ref{logowanie}) i jest Twórcą zamykanego Projektu (\ref{tworzenie_projektu})}
	% Post-conditions
	\postcond{Projekt został zamknięty, Użytkownik nie jest już Twórcą Projektu, }
\end{usecase}



\begin{usecase}{Tworzenie Zgłoszenia}
\label{tworzenie_zgloszenia}
	% Opis przypadku użycia
	\desc{Opisuje proces, w którym Użytkownik będący Członkiem Projektu dodaje nowe Zgłoszenie do Projektu}
	% Priorytet
	\priority{7}
	% Aktorzy
	\actors{Członek Projektu, Osoba Odpowiedzialna}
	% Przebieg główny
	\begin{enumerate}
      \item Członek Projektu przegląda istniejące Projekty (\ref{przegladanie_projektow})
      \item Członek Projektu wybiera odpowiedni Projekt
      \item Członek Projektu uzupełnia szczegóły nowego Zgłoszenia
      \item Członek Projektu zatwierdza powstanie nowego Zgłoszenia
	\end{enumerate}
	% Przebiegi alternatywne
	\beginalternatives
	\alternative{Wybranie Osoby Odpowiedzialnej za Zgłoszenie}{Członek Projektu wybiera Osobę Odpowiedzialną za utworzone zgłoszenie spośród Członków Projektu (\ref{wyznaczenie_osoby_odpowiedzialnej})}
	% Diagram
	\diagram{tworzenie_zgloszenia.pdf}{0.9} % drugi argument to szerokość obrazu (1.0 = szerokość strony)
	% Pre-conditions
	\precond{Użytkownik (Członek Projektu) jest zalogowany (\ref{logowanie}) i jest członkiem przynajmniej jednego Projektu}
	% Post-conditions
	\postcond{Powstało nowe Zgłoszenie w Projekcie, Członek Projektu staje się Twórca Zgłoszenia dla utworzonego Zgłoszenia}
\end{usecase}

% DONE
\begin{usecase}{Zamknięcie Zgłoszenia}
	% Opis przypadku użycia
	\desc{Twórca Zgłoszenia zamyka swoje Zgłoszenie}
	% Priorytet
	\priority{6}
	% Aktorzy
	\actors{Twórca Zgłoszenia}
	\begin{enumerate}
    \item Twórca Zgłoszenia przegląda istniejące Projekty (\ref{przegladanie_projektow})
    \item Twórca Zgłoszenia wybiera Projekt
    \item Twórca Zgłoszenia wybiera Zgłoszenie
    \item Twórca Zgłoszenia wydaje polecenie zamknięcia zgłoszenia
    \end{enumerate}
	% Diagram
	\diagram{zamkniecie_zgloszenia.pdf}{0.9} % drugi argument to szerokość obrazu (1.0 = szerokość strony)
	% Pre-conditions
    \precond{Użytkownik jest zalogowany (\ref{logowanie}) i jest Twórcą zamykanego Zgłoszenia (\ref{tworzenie_zgloszenia})}
	% Post-conditions
	\postcond{Zgłoszenie zostało zamknięte, Użytkownik nie jest już Twórcą Zgłoszenia}
\end{usecase}

% DONE
\begin{usecase}{Zmiana widoczności Zgłoszenia}
	% Opis przypadku użycia
	\desc{Twórca Zgłoszenia zmienia widoczność swojego Zgłoszenia}
	% Priorytet
	\priority{4}
	% Aktorzy
	\actors{Twórca Zgłoszenia}
    % Przebieg główny
	\begin{enumerate}
    \item Twórca Zgłoszenia przegląda istniejące Projekty (\ref{przegladanie_projektow})
    \item Twórca Zgłoszenia wybiera Projekt
    \item Twórca Zgłoszenia wybiera Zgłoszenie
    \item Twórca Zgłoszenia wydaje polecenia ukrycia/odkrycia Zgłoszenia
	\end{enumerate}
	% Diagram
	\diagram{zmiana_widocznosci_zgloszenia.pdf}{0.9} % drugi argument to szerokość obrazu (1.0 = szerokość strony)
	% Pre-conditions
	\precond{Użytkownik jest zalogowany (\ref{logowanie}) i jest Twórcą edytowanego Zgłoszenia (\ref{tworzenie_zgloszenia})}
	% Post-conditions
	\postcond{Widoczność Zgłoszenia została zmieniona}
\end{usecase}

% DONE
\begin{usecase}{Edycja Zgłoszenia}
\label{edycja_zgloszenia}
	% Opis przypadku użycia
	\desc{Twórca Zgłoszenia edytuje podstawowe informacje o Zgłoszeniu}
	% Priorytet
	\priority{6}
	% Aktorzy
	\actors{Twórca Zgłoszenia}
    % Przebieg główny
	\begin{enumerate}
	\item Twórca Zgłoszenia przegląda istniejące Projekty (\ref{przegladanie_projektow})
    \item Twórca Zgłoszenia wybiera Projekt
    \item Twórca Zgłoszenia wybiera Zgłoszenie
    \item Twórca Zgłoszenia wydaje polecenie edycji Zgłoszenia
    \item Twórca Zgłoszenia edytuje formularz
    \item Twórca Zgłoszenia zatwierdza zmiany
	\end{enumerate}
	% Diagram
	\diagram{edycja_zgloszenia.pdf}{0.9} % drugi argument to szerokość obrazu (1.0 = szerokość strony)
	% Pre-conditions
	\precond{Użytkownik jest zalogowany (\ref{logowanie}), należy do przynajmniej jednego Projektu i jest Twórcą przynajmniej jednego Zgłoszenia (\ref{tworzenie_zgloszenia})}
	% Post-conditions
	\postcond{Właściwości Zgłoszenia zostały zmienione}
\end{usecase}

% DONE
\begin{usecase}{Wyznaczenie Osoby Odpowiedzialnej}
\label{wyznaczenie_osoby_odpowiedzialnej}
	% Opis przypadku użycia
	\desc{Twórca Zgłoszenia wyznacza Użytkownika jako Odbiorcę Zgłoszenia dla danego Zgłoszenia}
	% Priorytet
	\priority{6}
	% Aktorzy
	\actors{Twórca Zgłoszenia, Osoba Odpowiedzialna, Użytkownik}
	% Przebieg główny
	\begin{enumerate}
    \item Twórca Zgłoszenia przegląda istniejące Projekty (\ref{przegladanie_projektow})
  	\item Twórca Zgłoszenia wybiera Projekt
  	\item Twórca Zgłoszenia wybiera Zgłoszenie
  	\item Twórca Zgłoszenia wydaje polecenie edycji Zgłoszenia (\ref{edycja_zgloszenia})
  	\item Twórca Zgłoszenia wybiera z listy Użytkowników Osobę Odpowiedzialną
  	\item Twórca Zgłoszenia zatwierdza zmiany
	\end{enumerate}
	% Diagram
	\diagram{wyznaczenie_osoby_odpowiedzialnej.pdf}{0.9} % drugi argument to szerokość obrazu (1.0 = szerokość strony)
	% Pre-conditions
	\precond{Użytkownik jest zalogowany (\ref{logowanie}), należy do przynajmniej jednego Projektu i jest Twórcą przynajmniej jednego Zgłoszenia (\ref{tworzenie_zgloszenia})}
	% Post-conditions
	\postcond{Do zgłoszenia została przypisana osoba odpowiedzialna}
    
\end{usecase}

% DONE
\begin{usecase}{Edycja Grafu}
	% Opis przypadku użycia
	\desc{Twórca Projektu planuje kolejne etapy Projektu w formie Zgłoszeń, przechowywanych w postaci Grafu}
	% Priorytet
	\priority{8}
	% Aktorzy
	\actors{Twórca Projektu}
	% Przebieg główny
	\begin{enumerate}
    \item Twórca Projektu przegląda istniejące Projekty (\ref{przegladanie_projektow})
    \item Twórca Projektu wybiera Projekt
    \item Twórca Projektu wybiera istniejące Zgłoszenie
    \item Twórca Projektu wyznacza rodzica Zgłoszenia lub usuwa istniejące dzieci
	\end{enumerate}
	% Przebiegi alternatywne
	% Diagram
	\diagram{tworzenie_grafu.pdf}{0.9} % drugi argument to szerokość obrazu (1.0 = szerokość strony)
	% Pre-conditions
	\precond{Użytkownik jest zalogowany (\ref{logowanie})  i jest Twórcą przynajmniej jednego Projektu (\ref{tworzenie_projektu})}
	% Post-conditions
	\postcond{Struktura Grafu została zmieniona}
\end{usecase}

\begin{usecase}{Ustawienie szczegółowych uprawnień}
	% Opis przypadku użycia
	\desc{Twórca Projektu ustawia szczegółowe uprawnienia innym Członkom Projektu}
	% Przebieg główny
	% Priorytet
	\priority{5}
	% Aktorzy
	\actors{Twórca Projektu, Członek Projektu}
	\begin{enumerate}
    \item Użytkownik wchodzi w sekcję ‘swoje projekty’
    \item Użytkownik wybiera odpowiedni projekt
    \item Użytkownik wybiera opcję ‘zarządzaj Członkami Projektu’
    \item Użytkownik wybiera odpowiedniego Członka Projektu
    \item Użytkownik zmienia uprawnienia wybranego Członka Projektu
	\end{enumerate}
	% Przebiegi alternatywne
	% Diagram
	\diagram{ustawienia_szczegolowych_uprawnien.pdf}{1.0} % drugi argument to szerokość obrazu (1.0 = szerokość strony)
	% Pre-conditions
	\precond{Użytkownik jest zalogowany (\ref{logowanie})  i jest Twórcą przynajmniej jednego Projektu (\ref{tworzenie_projektu})}
	% Post-conditions
	\postcond{Uprawnienia co najmniej jednego użytkownika zostały zmienione}
\end{usecase}

\begin{usecase}{Tworzenie projektu z wieloma zgłoszeniami}
	% Opis przypadku użycia
	\desc{Użytkownik tworzy projekty zawierające wiele Zgłoszeń}
	% Priorytet
	\priority{4}
	% Aktorzy
	\actors{Użytkownik, Twórca Projektu}
	% Przebieg główny
	\begin{enumerate}
    \item Użytkownik wybiera opcję ‘utwórz nowy Projekt’
    \item Użytkownik wypełnia odpowiednia informacje o Nowym Projekcie
    \item Użytkownik dodaje innych użytkowników jako Członków Projektu
    \item Użytkownik tworzy Zgłoszenia w danym projekcie.
	\end{enumerate}
	% Przebiegi alternatywne
	% Diagram
	\diagram{tworzenie_projektu_z_wieloma_zgloszeniami.pdf}{1.0} % drugi argument to szerokość obrazu (1.0 = szerokość strony)
	% Pre-conditions
	\precond{Użytkownik jest zalogowany (\ref{logowanie})}
	% Post-conditions
	\postcond{Powstanie projekt z wieloma zgłoszeniami}
\end{usecase}

\begin{usecase}{Udzielenie informacji zwrotnej}
	% Opis przypadku użycia
	\desc{Odbiorca Zgłoszenia udziela informacji zwrotnej Osobie Odpowiedzialnej przez zmianę statusu Zgłoszenia.}
	% Priorytet
	\priority{5}
	% Aktorzy
	\actors{Odbiorca Zgłoszenia, Osoba Odpowiedzialna}
	% Przebieg główny
	\begin{enumerate}
    \item Odbiorca Zgłoszenia  loguje się
    \item Odbiorca Zgłoszenia wchodzi w sekcję "moje projekty"
    \item Odbiorca Zgłoszenia wybiera Projekt
    \item Odbiorca Zgłoszenia wybiera Zgłoszenie
    \item Odbiorca Zgłoszenia edytuje Zgłoszenie
    \item Odbiorca Zgłoszenia ustawia odpowiedni status Zgłoszenia i zamieszcza komentarz
    \item Odbiorca Zgłoszenia zatwierdza zmiany
	\end{enumerate}
	% Przebiegi alternatywne
	% Diagram
	\diagram{udzial_informacji_zwrotnej.pdf}{1.0} % drugi argument to szerokość obrazu (1.0 = szerokość strony)
	% Pre-conditions
	\precond{Odbiorca Zgłoszenia jest twórcą przynajmniej jednego Projektu (\ref{tworzenie_projektu})}
	% Post-conditions
	\postcond{Zmiana statusu}
\end{usecase}

\begin{usecase}{Aktualizacja zasobów}
	% Opis przypadku użycia
	\desc{Odbiorca Zgłoszenia aktualizuje Zasoby Zgłoszenia.}
	% Priorytet
	\priority{4}
	% Aktorzy
	\actors{Odbiorca Zgłoszenia}
	% Przebieg główny
	\begin{enumerate}
    \item Odbiorca Zgłoszenia wchodzi na stronę
    \item Ta osoba loguje się jako root
    \item Próbuje zmienić “propozycję”
    \item Jeżeli taka jest to wraca do punktu 3
    \item Zmiana się powiodła
	\end{enumerate}
	% Przebiegi alternatywne
	% Diagram
	\diagram{aktualizacja_zasobow.pdf}{1.0} % drugi argument to szerokość obrazu (1.0 = szerokość strony)
	% Pre-conditions
	\precond{Odbiorca Zgłoszenia jest twórcą przynajmniej jednego Projektu (\ref{tworzenie_projektu})}
	% Post-conditions
	\postcond{Zmiana zasobów zgłoszenia}
\end{usecase}

\begin{usecase}{Pobieranie załączników}
	% Opis przypadku użycia
	\desc{Członek Projektu pobiera załączniki dodane do Zgłoszeń}
	% Priorytet
	\priority{3}
	% Aktorzy
	\actors{Członek Projektu}
	% Przebieg główny
	\begin{enumerate}
    \item Osoba wchodzi na stronę
    \item Loguje się podając Imię Nazwisko i nr. PESEL
    \item Osoba wyszukuje zgłoszenie
    \item Pobiera załącznik.
    \item Jeżeli to jest Twórca projektu, jest opcja pobrać wszystkie załączniki
	\end{enumerate}
	% Przebiegi alternatywne
	% Diagram
	\diagram{pobieranie_zalacznikow.pdf}{1.0} % drugi argument to szerokość obrazu (1.0 = szerokość strony)
	% Pre-conditions
	\precond{Odbiorca Zgłoszenia jest twórcą przynajmniej jednego Projektu (\ref{tworzenie_projektu})}
	% Post-conditions
	\postcond{Zmiana zasobów zgłoszenia}
\end{usecase}

\begin{usecase}{Przeszukiwanie zgłoszeń}
	% Opis przypadku użycia
	\desc{Członek Projektu przeszukuje Zgłoszenia w Projekcie.}
	% Priorytet
	\priority{4}
	% Aktorzy
	\actors{Użytkownik}
	% Przebieg główny
	\begin{enumerate}
    \item Użytkownik loguje się
    \item W polu “Znajdź zgłoszenie” wpisuje kluczowe słowa
    \item Przeglądarka wyświetla znalezione zgłoszenia
    \item Jeżeli nie znalazł nic wyświetla komunikat “Takich zgłoszeń nie ma”
	\end{enumerate}
	% Przebiegi alternatywne
	% Diagram
	\diagram{przeszukiwanie_zgloszen.pdf}{1.0} % drugi argument to szerokość obrazu (1.0 = szerokość strony)
	% Pre-conditions
	\precond{Odbiorca Zgłoszenia jest twórcą przynajmniej jednego Projektu (\ref{tworzenie_projektu})}
	% Post-conditions
	\postcond{Uzyskanie odpowiednich zgłoszeń lub informacji że ich brak}
\end{usecase}

\begin{usecase}{Wysyłanie informacji o zamknięciu projektu, "Odbiorca-Twórca"}
	% Opis przypadku użycia
	\desc{Odbiorca Zgłoszenia wysyłam Twórcy Zgłoszenia informacje, że Zgłoszenie jest do zamknięcia. (ja bym dał: Odbiorca Zgłoszenia za Zgłoszenie zmienia jego status.)}
	% Priorytet
	\priority{5}
	% Aktorzy
	\actors{Odbiorca Zgłoszenia, Twórca Zgłoszenia}
	% Przebieg główny
	\begin{enumerate}
    \item Odbiorca Zgłoszenia loguje się do systemu
    \item Odbiorca Zgłoszenia wchodzi do sekcji zgłoszenia
    \item W tym zgłoszenie jest podany adres @mail
    \item Odbiorca Zgłoszenia wysyła mail
	\end{enumerate}
	% Przebiegi alternatywne
	% Diagram
	\diagram{odbiorca_tworca.pdf}{1.0} % drugi argument to szerokość obrazu (1.0 = szerokość strony)
	% Pre-conditions
	\precond{Odbiorca i twórca muszą należeć do co najmniej jednego projektu.}
	% Post-conditions
	\postcond{Uzyskanie odpowiednich zgłoszeń lub informacji, że jest ich brak}
\end{usecase}

\begin{usecase}{Wysłanie informacji o zamknięciu projektu, "Użytkownik-Twórca"}
	% Opis przypadku użycia
	\desc{Członek Projektu wysyła Twórcy Projektu informacje, że  projekt jest do zamknięcia.}
	% Priorytet
	\priority{4}
	% Aktorzy
	\actors{Członek Projektu, Twórca Projektu}
	% Przebieg główny
	\begin{enumerate}
    \item Członek projektu wchodzi na stronę
    \item Podaje Imię, Nazwisko i identyfikator
    \item Naciska na przycisk “Projekt jest do zamknięcia”
    \item Wiadomość przychodzi do Twórcy projektu
	\end{enumerate}
	% Przebiegi alternatywne
	% Diagram
	\diagram{uzytkownik_tworca.pdf}{1.0} % drugi argument to szerokość obrazu (1.0 = szerokość strony)
	% Pre-conditions
	\precond{Odbiorca i twórca muszą należeć do co najmniej jednego projektu.}
	% Post-conditions
	\postcond{Uzyskanie odpowiednich zgłoszeń lub informacji że jest ich brak}
\end{usecase}