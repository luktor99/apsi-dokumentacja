\begin{usecase}{Zarządzanie Systemem}
	% Opis przypadku użycia
	\desc{Ten ogólny przypadek użycia opisuje jak Administrator może zarządzać Systemem.}
    % Priorytet
	\priority{9}
	% Aktorzy
	\actors{Administrator}
	% Przebieg główny
	\main{Ten przypadek użycia zaczyna się gdy Administrator wejdzie do panelu Administratora}
	\begin{enumerate}
		\item Administrator robi backup danych
		\item Administrator wykonuje przywracanie systemu
		\item Administrator przegląda logi systemu
        \item Administrator przeprwowadza rekonfigurację systemu
        \item Administrator zarządza kontem Użytkownika
	\end{enumerate}
	% Przebiegi alternatywne
	\beginalternatives
	\alternative{Administrator próbował wykonać przywracanie systemu w niedozwolonym momencie}{ System nie pozwoli Administratorowi wykonać takiej opracji w innym momencie niż ten ściśle określony w kontrakcie}
	%\alternative{Dodatkowa pozycja}{W ten sposób można dodawać kolejne przebiegi alternatywne \ldots}
	% Diagram
	\diagram{zarzadzanie_systemem.pdf}{1} % drugi argument to szerokość obrazu (1.0 = szerokość strony)
	% Pre-conditions
	\precond{Administrator jest zalogowany w systemie. }
	% Post-conditions
	\postcond{Jeśli przypadek użycia został pomyślnie zrealizowany, stan danego konta użytkownika został zmieniony.}
    % Referencje
	\references{\rf{5} \rf{6} \rf{7} \rf{8} \rf{9} \rf{10} \rnf{2} \rnf{4} \rnf{5}}
\end{usecase}

\begin{usecase}{Zarządzanie kontem Użytkownika}
	% Opis przypadku użycia
	\desc{Ten ogólny przypadek użycia opisuje jak Administrator zarządza kontem użytkownika.}
    % Priorytet
	\priority{9}
	% Aktorzy
	\actors{Administrator}
	% Przebieg główny
	\main{Ten przypadek użycia zaczyna się gdy Administrator wejdzie do panelu Administratora}
	\begin{enumerate}
		\item Administrator Zatwierdza konto nowo-zarejestrowanego Użytkownika
		\item Administrator zmienia uprawnienia użytkownika
		\item Administrator usuwa konto użytkownika
	\end{enumerate}
	% Przebiegi alternatywne
	\beginalternatives
	\alternative{Administrator próbował zarządać kontem innego Administratora}{ Jeśli Administrator próbował zarządzać kontem innego Administratora, System odrzuca zmiany co kończy ten przypadek użycia}
	%\alternative{Dodatkowa pozycja}{W ten sposób można dodawać kolejne przebiegi alternatywne \ldots}
	% Diagram
	\diagram{zarzadzanie_kontem.pdf}{1} % drugi argument to szerokość obrazu (1.0 = szerokość strony)
	% Pre-conditions
	\precond{Administrator jest zalogowany w Systemie. }
	% Post-conditions
	\postcond{Jeśli przypadek użycia został pomyślnie zrealizowany, stan danego konta użytkownika został zmieniony.}
    % Referencje
	\references{\rf{5} \rf{6} \rf{7}}
\end{usecase}

\begin{usecase}{Zatwierdzanie konta}
	% Opis przypadku użycia
	\desc{Ten przypadek użycia opisuje proces zatwierdzania kont nowo-zarejestrowanych użytkowników w systemie Grapics.}
    % Priorytet
	\priority{10}
	% Aktorzy
	\actors{Administrator}
	% Przebieg główny
	\main{Ten przypadek użycia zaczyna się gdy Administrator otrzymał powiadomienie o nowo-utworzonych kontach się w systemie.}
	\begin{enumerate}
		\item System wyświetla nowych użytkowników w panelu Administratorskim
        \item Administrator wybiera jedno z kont
		\item Administrator weryfikuje i zatwierdza wybrane konto
	\end{enumerate}
	% Przebiegi alternatywne
	\beginalternatives
	\alternative{Osoba nieupowazniona próbowała założyć konto}{Jeśli osoba nieupoważniona, to znaczy taka, która nie jest klientem, próbowała założyć konto w systemie, konto zostaje odrzucone przez administratora}
	%\alternative{Dodatkowa pozycja}{W ten sposób można dodawać kolejne przebiegi alternatywne \ldots}
	% Diagram
	\diagram{zatwierdzanie_konta.pdf}{1} % drugi argument to szerokość obrazu (1.0 = szerokość strony)
    % Diagram sekwencji
	\sekwencja{zatwierdzenie_konta_sekwencja.pdf}{1}
	% Diagram klas
	\klasa{zatwierdzenie_konta_klas.pdf}{1}
	% Pre-conditions
	\precond{Administrator jest zalogowany w Systemie oraz istnieją niezatwierdzone konta w Systemie.}
	% Post-conditions
	\postcond{Jeśli przypadek użycia został pomyślnie zrealizowany, konto jest potwierdzone przez Administratora w Systemie i może on posłużyć do zalogowania się. W przeciwnej sytuacji stan systemu pozostaje niezmieniony.}
    % Referencje
	\references{\rf{5}}
\end{usecase}

\begin{usecase}{Usuwanie konta}
	% Opis przypadku użycia
	\desc{Ten przypadek użycia opisuje proces usuwania kont użytkowników w systemie Grapics.}
    % Priorytet
	\priority{8}
	% Aktorzy
	\actors{Administrator}
	% Przebieg główny
	\main{Ten przypadek użycia zaczyna się gdy Administrator otrzymał powiadomienie o wygaścięciu ważności konta w Systemie Grapics.}
	\begin{enumerate}
		\item System wyświetla użytkowników w panelu Administratorskim
        \item Administrator wybiera jedno z kont
		\item Administrator weryfikuje i usuwa wybrane konto
	\end{enumerate}
	% Przebiegi alternatywne
	%\alternative{Dodatkowa pozycja}{W ten sposób można dodawać kolejne przebiegi alternatywne \ldots}
	% Diagram
	\diagram{usuwanie_konta.pdf}{1} % drugi argument to szerokość obrazu (1.0 = szerokość strony)
    % Diagram sekwencji
	\sekwencja{usuwanie_konta_sekwencja.pdf}{1}
	% Diagram klas
	\klasa{usuwanie_konta_zmiana_uprawnien_klas.pdf}{1}
	% Pre-conditions
	\precond{Administrator jest zalogowany w Systemie oraz dostał powiadomienie o wygaśnięciu wazności konta użytkownika.}
	% Post-conditions
	\postcond{Jeśli przypadek użycia został pomyślnie zrealizowany, konto jest usunięte przez Administratora w Systemie i nie może ono posłużyć do zalogowania się. W przeciwnej sytuacji stan systemu pozostaje niezmieniony.}
    % Referencje
	\references{\rf{7}}
\end{usecase}

\begin{usecase}{Zmiana uprawnień}
	% Opis przypadku użycia
	\desc{Ten przypadek użycia opisuje proces zmieniania uprawnież kont użytkowników w systemie Grapics.}
    % Priorytet
	\priority{8}
	% Aktorzy
	\actors{Administrator}
	% Przebieg główny
	\main{Ten przypadek użycia zaczyna się gdy Administrator otrzymał powiadomienie o potrzebie zmiany uprawnień konta w Systemie Grapics.}
	\begin{enumerate}
		\item System wyświetla użytkowników w panelu Administratorskim
        \item Administrator wybiera jedno z kont
		\item Administrator weryfikuje i zmienia uprawnienia wybranengo konta
	\end{enumerate}
	% Przebiegi alternatywne
	%\alternative{Dodatkowa pozycja}{W ten sposób można dodawać kolejne przebiegi alternatywne \ldots}
	% Diagram
	\diagram{zmiana_uprawnien.pdf}{1} % drugi argument to szerokość obrazu (1.0 = szerokość strony)
    % Diagram sekwencji
	\sekwencja{zmiana_uprawnien_sekwencja.pdf}{1}
	% Diagram klas
	\klasa{usuwanie_konta_zmiana_uprawnien_klas.pdf}{1}
	% Pre-conditions
	\precond{Administrator jest zalogowany w Systemie oraz dostał powiadomienie o potrzebie zmiany uprawnień danego konta użytkownika.}
	% Post-conditions
	\postcond{Jeśli przypadek użycia został pomyślnie zrealizowany, uprawnienia danego konta zostały zmienione. W przeciwnej sytuacji stan systemu pozostaje niezmieniony.}
    % Referencje
	\references{\rf{6}}
\end{usecase}

\begin{usecase}{Backup danych}
	% Opis przypadku użycia
	\desc{Ten przypadek użycia opisuje proces robienia backupu danych przez Administratora.}
    % Priorytet
	\priority{10}
	% Aktorzy
	\actors{Administrator}
	% Przebieg główny
	\main{Ten przypadek użycia zaczyna się gdy Administrator otrzymał powiadomienie o potrzebie zrobienia backupu danych w Systemie Grapics.}
	\begin{enumerate}
		\item Administrator wybiera opcję "Backup danych" w panelu Administratorskim
        \item Administrator zaznacza odpowiednie opcje
		\item Administrator czeka aż proces dobiegnie końca
	\end{enumerate}
	% Przebiegi alternatywne
	%\alternative{Dodatkowa pozycja}{W ten sposób można dodawać kolejne przebiegi alternatywne \ldots}
	% Diagram
	\diagram{backup_danych.pdf}{1} % drugi argument to szerokość obrazu (1.0 = szerokość strony)
    % Diagram sekwencji
	\sekwencja{backup_sekwencja.pdf}{1}
	% Diagram klas
	\klasa{backup_klas.pdf}{1}
	% Pre-conditions
	\precond{Administrator jest zalogowany w Systemie oraz dostał powiadomienie o potrzebie wykonania Backupu.}
	% Post-conditions
	\postcond{Jeśli przypadek użycia został pomyślnie zrealizowany, backup Systemu został wykonany. W przeciwnej sytuacji stan systemu pozostaje niezmieniony.}
    % Referencje
	\references{\rf{8} \rnf{2}}
\end{usecase}

\begin{usecase}{Przywracanie Systemu}
	% Opis przypadku użycia
	\desc{Ten przypadek użycia opisuje proces przywracania Systemu przez Administratora.}
    % Priorytet
	\priority{10}
	% Aktorzy
	\actors{Administrator}
	% Przebieg główny
	\main{Ten przypadek użycia zaczyna się gdy Administrator otrzymał powiadomienie o potrzebie przywrócenia Systemu Grapics.}
	\begin{enumerate}
		\item Administrator odnajduje odpowiednią wersję Systemu, którą należy przywrócić
        \item Administrator wybiera opcję "Przywracanie Systemu" z panelu Administratora
        \item Administrator zaznacza odpowiednie opcje
		\item Administrator czeka aż proces dobiegnie końca
	\end{enumerate}
	% Przebiegi alternatywne
    \beginalternatives
	\alternative{Administrator próbował wykonać przywracanie systemu w niedozwolonym momencie}{ System nie pozwoli Administratorowi wykonać takiej opracji w innym momencie niż ten ściśle określony w kontrakcie}
	%\alternative{Dodatkowa pozycja}{W ten sposób można dodawać kolejne przebiegi alternatywne \ldots}
	% Diagram
	\diagram{przywracanie_systemu.pdf}{1} % drugi argument to szerokość obrazu (1.0 = szerokość strony)
    % Diagram sekwencji
	\sekwencja{przywracanie_systemu_sekwencja.pdf}{1}
	% Diagram klas
	\klasa{przywracanie_systemu_klas.pdf}{1}
	% Pre-conditions
	\precond{Administrator jest zalogowany w Systemie oraz dostał powiadomienie o potrzebie przywrócenia Systemu.}
	% Post-conditions
	\postcond{Jeśli przypadek użycia został pomyślnie zrealizowany, odpowiednia wersja Systemu została przywrócona. W przeciwnej sytuacji stan systemu pozostaje niezmieniony.}
    % Referencje
	\references{\rf{10}}
\end{usecase}

\begin{usecase}{Przeglądanie logów Systemu}
	% Opis przypadku użycia
	\desc{Ten przypadek użycia opisuje przeglądania logów systemu przez Administratora.}
    % Priorytet
	\priority{7}
	% Aktorzy
	\actors{Administrator}
	% Przebieg główny
	\main{Ten przypadek użycia zaczyna się gdy Administrator zamierza przejrzeć logi w Systemie Grapics.}
	\begin{enumerate}
		\item Administrator wybiera opcję "Logi" w panelu Administratora
        \item Administrator zaznacza odpowiednie opcje
		\item Administrator czeka aż zbieranie logów dobiegnie końca
        \item Administrator otrzymuje logi w formacie .txt
	\end{enumerate}
	% Przebiegi alternatywne
	%\alternative{Dodatkowa pozycja}{W ten sposób można dodawać kolejne przebiegi alternatywne \ldots}
	% Diagram
	\diagram{przegladanie_logow.pdf}{1} % drugi argument to szerokość obrazu (1.0 = szerokość strony)
    % Diagram sekwencji
	\sekwencja{przegladanie_logow_sekwencja.pdf}{1}
	% Diagram klas
	\klasa{przegladanie_logow_klas.pdf}{1}
	% Pre-conditions
	\precond{Administrator jest zalogowany w Systemie.}
	% Post-conditions
	\postcond{Jeśli przypadek użycia został pomyślnie zrealizowany, logi Systemu zostały wygenerowane do pliku .txt. W przeciwnej sytuacji stan systemu pozostaje niezmieniony.}
    % Referencje
	\references{\rf{9}}
\end{usecase}

\begin{usecase}{Rekonfiguracja Systemu}
	% Opis przypadku użycia
	\desc{Ten przypadek użycia opisuje proces rekonfiguracji Systemu przez Administratora.}
    % Priorytet
	\priority{9}
	% Aktorzy
	\actors{Administrator}
	% Przebieg główny
	\main{Ten przypadek użycia zaczyna się gdy Administrator otrzymał powiadomienie o potrzebie rekonfiguracji Systemu Grapics.}
	\begin{enumerate}
		\item Administrator wybiera opcję "Rekonfiguracja Systemu" w panelu Administratorskim
        \item Administrator zaznacza odpowiednie opcje
		\item Administrator czeka aż proces dobiegnie końca
	\end{enumerate}
	% Przebiegi alternatywne
    \beginalternatives
	\alternative{Administrator próbował wykonać Rekonfiguracje Systemu w niedozwolonym momencie}{ System nie pozwoli Administratorowi wykonać takiej opracji w innym momencie niż ten ściśle określony w kontrakcie}
	%\alternative{Dodatkowa pozycja}{W ten sposób można dodawać kolejne przebiegi alternatywne \ldots}
	% Diagram
	\diagram{rekonfiguracja_systemu.pdf}{1} % drugi argument to szerokość obrazu (1.0 = szerokość strony)
    % Diagram sekwencji
	\sekwencja{rekonfiguracja_sekwencja.pdf}{1}
	% Diagram klas
	\klasa{rekonfiguracja_klas.pdf}{1}
	% Pre-conditions
	\precond{Administrator jest zalogowany w Systemie oraz dostał powiadomienie o potrzebie rekonfiguracji Systemu.}
	% Post-conditions
	\postcond{Jeśli przypadek użycia został pomyślnie zrealizowany, rekonfiguracja Systemu została wykonana. W przeciwnej sytuacji stan systemu pozostaje niezmieniony.}
    % Referencje
	\references{\rf{10}}
\end{usecase}