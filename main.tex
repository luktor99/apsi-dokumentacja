\documentclass[12pt, oneside, final]{report}
\usepackage{geometry}
\geometry{a4paper, left=20mm, right=20mm, top=25mm, bottom=25mm}
\usepackage[utf8]{inputenc}
\usepackage{t1enc}
\usepackage[MeX]{polski}
\usepackage{graphicx}
\usepackage{amsmath}
\usepackage{amssymb}
\usepackage{mathtools}
\usepackage{indentfirst}
\usepackage{pdfpages}
\usepackage{xcolor}
\usepackage{placeins} % provides \FloatBarrier
\usepackage{tikz}
\usetikzlibrary{positioning,shapes,arrows,calc,decorations.markings,shadows}
\usepackage[hidelinks]{hyperref}

% Czcionka
\usepackage{charter}

% Pojedyncze elementy na górze stron
\makeatletter
\setlength{\@fptop}{0pt}
\makeatother

% Styl tytułowania rozdziałów:
\usepackage{titlesec}
%\titleformat{\chapter}{\normalfont\huge}{\bf\thechapter.}{20pt}{\huge\bf}
% Styl tytułowania rozdziałów:
\titleformat{\chapter}[display] 
{\centering\normalfont\huge\bfseries}{\centering\chaptertitlename\ \thechapter.}{0.5em}{}
\titlespacing{\chapter}{0em}{0em}{2em}

% dodatkowe kropki po numerach rozdziałów, sekcji itd
\usepackage{titlesec}
\titlelabel{\thetitle.\quad}

% Dodatkowe odstępy w tabelach (do UseCaseow)
\usepackage{array}
\setlength\extrarowheight{4pt}


% Szablon do Use Caseów
\newcounter{UseCaseCounter}
\setcounter{UseCaseCounter}{1}
\newcounter{AlternativeCounter}
\setcounter{AlternativeCounter}{1}

\definecolor{colorUC}{RGB}{42,161,152}
\newcommand{\usecasebox}[1]{\colorbox{colorUC}{\textcolor{white}{#1}}}

\newcommand{\header}[1]{\subsection[(PU\arabic{UseCaseCounter}) #1]{\usecasebox{PU\arabic{UseCaseCounter}} #1}}

\newcommand{\desc}[1]{\subsubsection{Opis przypadku użycia}
	\noindent #1}

\newcommand{\main}[1]{\subsubsection{Podstawowy przebieg}
	\noindent #1}

\newcommand{\beginalternatives}{\subsubsection{Przebiegi alternatywne}\noindent}
\newcommand{\alternative}[2]{\noindent\begin{tabular}{|p{\textwidth}|}
		\hline
		\textbf{PU\arabic{UseCaseCounter}.\Alph{AlternativeCounter} #1}\\
		#2\\
		\hline
	\end{tabular}\vspace{0.5em}\stepcounter{AlternativeCounter}}

\newcommand{\diagram}[2]{\vspace{-0.5em}\subsubsection{Diagram przypadku użycia}\noindent
	\begin{figure}[!ht]
		\centering
		\includegraphics[width=#2\textwidth]{diagrams/#1}
	\end{figure}
	\FloatBarrier}

\newcommand{\sekwencja}[2]{\vspace{-0.5em}\subsubsection{Diagram sekwencji}\noindent
	\begin{figure}[!ht]
		\centering
		\includegraphics[width=#2\textwidth]{diagrams/#1}
	\end{figure}
	\FloatBarrier}
    
\newcommand{\klasa}[2]{\vspace{-0.5em}\subsubsection{Diagram klas}\noindent
	\begin{figure}[!ht]
		\centering
		\includegraphics[width=#2\textwidth]{diagrams/#1}
	\end{figure}
	\FloatBarrier}

\newcommand{\precond}[1]{\subsubsection{Warunki wstępne}
	\noindent #1}

\newcommand{\postcond}[1]{\subsubsection{Warunki końcowe}
	\noindent #1}

\newcommand{\references}[1]{\subsubsection{Referencje}
	\noindent #1}

\definecolor{xcolorFR}{RGB}{108,113,196}
\definecolor{xcolorNFR}{RGB}{220,50,47}
\newcommand{\xfuncreqbox}[1]{\colorbox{xcolorFR}{\textcolor{white}{#1}}}
\newcommand{\xnonfuncreqbox}[1]{\colorbox{xcolorNFR}{\textcolor{white}{#1}}}
\newcommand{\rf}[1]{\hyperref[F#1]{\xfuncreqbox{F#1}}}
\newcommand{\rnf}[1]{\hyperref[NF#1]{\xnonfuncreqbox{NF#1}}}

\newcommand{\actors}[1]{\subsubsection{Aktorzy: \textnormal{#1}}}
\newcommand{\priority}[1]{\subsubsection{Priorytet: \textnormal{#1}}}


\definecolor{colorTODO}{RGB}{255,0,255}
\definecolor{colorCOMMENT}{RGB}{255,120,0}
\newcommand{\todo}{\colorbox{colorTODO}{\textcolor{white}{\textbf{TODO}}}}
\newcommand{\comment}[1]{\textcolor{colorCOMMENT}{#1}}

\newenvironment{usecase}{\setcounter{AlternativeCounter}{1}\header}{\stepcounter{UseCaseCounter}}
% Szablon do wymagan funkc. i niefunkc.
\newcounter{FuncReqCounter}
\setcounter{FuncReqCounter}{1}
\newcounter{NonFuncReqCounter}
\setcounter{NonFuncReqCounter}{1}

\definecolor{colorFR}{RGB}{108,113,196}
\definecolor{colorNFR}{RGB}{220,50,47}
\newcommand{\funcreqbox}[1]{\colorbox{colorFR}{\textcolor{white}{#1}}}
\newcommand{\nonfuncreqbox}[1]{\colorbox{colorNFR}{\textcolor{white}{#1}}}

\newcommand{\headerFR}[1]{\subsection[(F\arabic{FuncReqCounter}) #1]{\funcreqbox{F\arabic{FuncReqCounter}} #1}}
\newcommand{\headerNFR}[1]{\subsection[(NF\arabic{NonFuncReqCounter}) #1]{\nonfuncreqbox{NF\arabic{NonFuncReqCounter}} #1}}
\newcommand{\prioritytable}[2]{\begin{tabular}[ht!]{|l|c|}
		\hline
		Priorytet realizacji: & #1\\
		\hline
		Złożoność realizacji: & #2\\
		\hline
\end{tabular}}

\newcommand{\funcreq}[3]{\headerFR{#1} \prioritytable{#2}{#3} \stepcounter{FuncReqCounter}}
\newcommand{\nonfuncreq}[3]{\headerNFR{#1} \stepcounter{NonFuncReqCounter}}
% Szablon do diagramów sekwencji
\newcounter{SequenceCounter}
\setcounter{SequenceCounter}{1}

\newcommand{\headerSQ}[1]{\subsection{#1}}

\newcommand{\diagramSQ}[2]{\vspace{-0.5em}\subsubsection{}\noindent
	\begin{figure}[!ht]
		\centering
		\includegraphics[width=#2\textwidth]{diagrams/#1}
	\end{figure}
	\FloatBarrier}

\newenvironment{sequence}{\headerSQ}{\stepcounter{SequenceCounter}}
%\newcommand{\sequnece}[3]{\headerSQ{#1} \diagramSQ{#2}{#3} \stepcounter{SequenceCounter}}
% Szablon do diagramów klas
\newcounter{ClassCounter}
\setcounter{ClassCounter}{1}

\newcommand{\headerCD}[1]{\subsection{#1}}

\newcommand{\diagramCD}[2]{\vspace{-0.5em}\subsubsection{Diagram klas}\noindent
	\begin{figure}[!ht]
		\centering
		\includegraphics[width=#2\textwidth]{diagrams/#1}
	\end{figure}
	\FloatBarrier}

\newenvironment{class}{\headerSQ}{\stepcounter{ClassCounter}}
% Szablon do diagramów komponentów
\newcounter{ComponentCounter}
\setcounter{ComponentCounter}{1}

\newcommand{\headerC}[1]{\subsection{#1}}

\newcommand{\diagramC}[2]{\vspace{-0.5em}\subsubsection{}\noindent
	\begin{figure}[!ht]
		\centering
		\includegraphics[width=#2\textwidth]{diagrams/#1}
	\end{figure}
	\FloatBarrier}

% W pliku t_database.tex też jest zdefiniowane srodowisko o takiej nazwie - popraw to, bo PDF się nie chce zbudować
%\newenvironment{component}{\headerSQ}{\stepcounter{ComponentCounter}}
% Szablon do diagramów bazy danych
\newcounter{DatabaseCounter}
\setcounter{DatabaseCounter}{1}

\newcommand{\headerD}[1]{\subsection{#1}}

\newcommand{\diagramD}[2]{\vspace{-0.5em}\subsubsection{}\noindent
	\begin{figure}[!ht]
		\centering
		\includegraphics[width=#2\textwidth]{diagrams/#1}
	\end{figure}
	\FloatBarrier}

\newenvironment{component}{\headerD}{\stepcounter{DatabaseCounter}}

\begin{document}

% Title page
\begin{titlepage}
\centering
\begin{figure}
	\centering
	\includegraphics[width=0.9\textwidth]{logo.pdf}
\end{figure}
\vspace*{100pt}
\LARGE{Projekt APSI}\\
\vspace{30pt}
\textsc{\Huge{System Grapics}}\\
\vspace{120pt}
\Large{Mikołaj Gapsa}\\
\Large{Łukasz Kilaszewski}\\
\Large{Vitali Kozak}\\
\Large{Piotr Rosłaniec}\\
\vfill
\large{Warszawa, 2017}
\end{titlepage}

\thispagestyle{empty}
\tableofcontents
\cleardoublepage

\chapter{Wprowadzenie}
\section{Opis projektu}
Projekt Grapics realizuje system informatyczny służący do zarządzania procesami biznesowymi w firmie realizującej zamówienia B2B na dobra intelektualne, w szczególności grafiki i fotografie.

\section{Procesy biznesowe}
Ze względu na zróżnicowane zadania jakie realizuje wykonawca (odbiorca systemu) proces realizacja zadania nie jest dobrze ustrukturyzowany. Utrudnia to monitorowania i nadzorowanie zadań, a także ich synchronizację między sobą w przypadkach, kiedy istnieją między nimi zależności.

Zmienność wymagań i ich niedoprecyzowanie sprawia, że konieczna jest ciągła komunikacja z klientem i pozyskiwanie informacji zwrotnej po wprowadzeniu znaczących zmian. Efektem każdej zmiany jest wydanie artefaktu w pewnej wersji. Artefakty wymagają nie tylko akceptacji klienta, ale również konkretnych osób z zespołu i przełożonego. 

Śledzenie i wersjonowanie artefaktów staje się konieczne w projektach, które przechodzą wiele iteracji nierzadko przy udziale podwykonawców. Artefakty mogą się znajdować w wielu różnych lokalizacjach, po stronie klienta, wykonawcy lub podwykonawców.

\section{Opis wprowadzonych pojęć}
W ramach opisu projektu wprowadzono następujące pojęcia:
\begin{itemize}
	\item \textit{Zgłoszenie} -- jednostka pracy o znaczeniu biznesowym,
	\item \textit{Artefakt} -- produkt realizacji \textit{Zgłoszenia},
	\item \textit{Projekt} -- zbiór \textit{Zgłoszeń},
	\item \textit{Repozytorium} -- lokalizacja przechowywująca \textit{Artefakty}.
\end{itemize}

\section{Aktorzy}
W systemie zidentyfikowano następujących aktorów:
\begin{itemize}
	\item \textit{Administrator} -- jednostka pracy o znaczeniu biznesowym,
	\item \textit{Użytkownik} -- podstawowy aktor, jego rola biznesowa jest określona za pomocą zestawu uprawnień dla poszczególnych projektów,
	\item \textit{Twórca projektu} (\texttt{<<extends>>} \textit{Użytkownik}) -- użytkownik sprawujący odpowiedzialność nad projektem,
	\item \textit{Twórca zgłoszenia} (\texttt{<<extends>>} \textit{Użytkownik}) -- użytkownik sprawujący odpowiedzialność nad zgłoszeniem,
	\item \textit{Osoba Odpowiedzialna} (\texttt{<<extends>>} \textit{Użytkownik}) -- użytkownik oddelegowany do zrealizowania zgłoszenia,
	\item \textit{Gość} -- osoba nie posiadająca dostępu do systemu.
\end{itemize}

\section{Moduły}
System podzielony jest na następujące moduły:
\begin{itemize}
	\item Moduł uwierzytelniania
	\item Moduł administracyjny
	\item Moduł główny (obsługi projektów)	
\end{itemize}
\subsection{Moduł uwierzytelniania}
Moduł odpowiedzialny za uwierztyelnianie użytkowników. Jego główne cele to umożliwienie gościom zakładanie kont oraz pozwolić użytkownikom zalogować się.
\subsection{Moduł administracyjny}
Moduł odpowiedzialny za wszelkie prace wykonywane przez Administratora Systemu. Uwzględniono tutaj wszystkie możliwości jakie można znaleźć w panelu Administratora.
\subsection{Moduł główny}
Moduł odpowiedzialny za szeroko rozumianą obsługę projektów. Moduł ten zawiera wszystkie operacje powiązane z tworzeniem projektów, zadań, a także zarządzania poszczególnymi elementami projektu.

\chapter{Specyfikacja sprzętu i oprogramowania podstawowego}
Do wdrożenia aplikacji konieczne jest zapewnienie odpowiedniego środowiska. Proponowana przez nas konfiguracja zakłada oddelegowanie odpowiedzalności za utrzymanie infrastruktury do chmury Amazon Web Services. Wymieniony usługodawcs zapewnia wysoki poziom dostępności i elastyczność konfiguracji przy relatywnie niskich kosztach. Stosowane usługi to m.in.

\begin{itemize}
	\item AWS Elastic Cloud Compute (EC2) - skalowalny najem maszyn wirtualnych w chmurze
	\item AWS Simple Storage Service (S3) - przestrzeń do magazynowania danych w pseudo-systemie plików
\end{itemize}

\section{Środowisko sprzętowe}
Poniżej zamieszczamy przykład środowiska sprzętowego dla wdrożenia z przyjętymi następującymi założeniami:
\begin{itemize}
\item Grupę użytkowników mieści się w przedziale 50-100 aktywnie wykorzystujących zasoby pracowników
\item System jest użytkowany średnio 20 godzin na dobę
\item Użytkownicy regularnie archiwizuja zasoby (transfer około 50GB miesięcznie)
\item System przechowuje pewną ilość zarchiwizowanych zasobów (1 TB)
\item Odbiorca systemu opłaca z góry połowę rocznego kosztu utrzymania infrastruktury
\end{itemize}

W takim wypadku konfiguracja sprzętowa wraz z kosztorysem przyjmuje następującą postać:

\begin{table}[]
\centering
\caption{Środowisko sprzętowe - kosztorys miesięczny }
\begin{tabular}{|l|l|l|l|}
\hline
%  http://calculator.s3.amazonaws.com/index.html#r=IAD&key=calc-8E7A0FE0-F8D1-4FF0-957B-0068FE48FEA5 
 & Typ usługi AWS & Konfiguracja & Miesięczny koszt \\ \hline
Serwer aplikacji 		 & EC2             & 4 CPU, 16 GiB RAM  & \$46.85                 \\ \hline
Baza danych &              EC2            &   4 CPU, 16 GiB RAM             &          \$46.85     \\ \hline
Archiwum zasobów &               S3           &   1 TB przestrzeni           &     \$23.56             \\ \hline
 Transfer danych &         Inter-regional transfer  &       50 GB       &         \$1.00         \\ \hline
 \multicolumn{3}{|l|}{Łącznie miesięcznie: } & \$118.26 \\ \hline
\end{tabular}
\end{table}



\section{System operacyjny}
Proponujemy implementacje i wdrożenie systemu w oparciu o rozwiązania otwarte, w tym także system operacyjny Linux. Proponowaną przez nas dystrybucja to CentOS 7. Klientom, którym zależy na technicznym i merytorycznym wsparciu proponujemy dostarczaną przez firmę RedHat wersję systemy klasy enterprise (RHEL 7)

\chapter{Specyfikacja technologii}
Aplikacja opiera się o szereg technologii, które dzięki wzajemnej współpracy dostarczają spójny i niezawodny system. Wykorzystane technologie można podzielić na następujące grupy:
\begin{itemize}
	\item technologie użyte po stronie serwera (back-end),
	\item technologie użyte w interfejsie użytkownika (front-end),
	\item technologie użyte do przechowywania danych.
\end{itemize}
\section{Technologie użyte po stronie serwera}
Obsługa żądań napływających od klientów, wydawanych za pośrednictwem interfejsu użytkownika, obsługiwana będzie przez dedykowaną aplikację napisaną w języku Java. Technologia ta znajduje szerokie zastosowanie w tego typu systemach ze względu na niezawodność, stosunkowy krótki czas rozwoju oprogramowania oraz dobrą skalowalność. Także w przypadku systemu Grapics, właśnie te czynniki wpłynęły na wybór tej technologii.

Do obsługi zarządzania projektem oprogramowania używane będzie narzędzie Apache Maven. Automatyzuje ono proces budowania aplikacji oraz upraszcza problem zarządzania kodem źródłowym i zależnościami.

Aby zapewnić spójność i standaryzację kodu źródłowego, używany będzie szkielet tworzenia aplikacji Spring, wraz z modułami:
\begin{itemize}
	\item Boot -- pozwala na szybkie tworzenie nowego projektu z określonymi zależnościami,
	\item Data -- obsługuje połączenie z bazą danych oraz realizuje wymagane transakcje,
	\item Web -- używany do tworzenia aplikacji webowych, w szczególności pozwala na wystawianie interfejsów REST, wykorzystywanych w systemie Grapics.
\end{itemize}
Do uruchamiania aplikacji po stronie serwera posłuży oprogramowanie Apache Tomcat, które pozwala na uruchamianie aplikacji stworzonych w technologii Spring w dedykowanym kontenerze. Zapewnia to niezawodność systemu, a w związku z tym umożliwia uzyskanie założonego czas dostępności.

\section{Technologie użyte w interfejsie użytkownika}
Interfejs użytkownika, w postaci aplikacji uruchamianej w przeglądarce, stworzony zostanie w oparciu o framework Angular 4. Jest to szkielet aplikacyjny oparty na języku JavaScript, który umożliwia tworzenie zaawansowanych aplikacji w krótkim czasie. 

Do uruchomienia stworzonej aplikacji internetowej użyte zostanie środowisko NodeJS. Zapewnia ono wysoką skalowalność tworzonych rozwiązań oraz dużą niezawodność.
\section{Technologie użyte do przechowywania danych}
Dane, które muszą być permanentnie przechowywane, składowane będą w bazie danych MongoDB. Ten system bazodanowy pozwala w sposób wydajny przechowywać wszelkie dokumenty, jakie mogą zostać powiązane z tworzonymi w systemie projektami i zgłoszeniami.
\chapter{Przypadki użycia}
\section{Moduł uwierzytelniania}
\begin{usecase}{Uwierzytelnianie użytkownika}
	% Opis przypadku użycia
	\desc{Ten ogólny przypadek użycia opisuje proces uwierzytelniania użytkowników w systemie Grapics.}
    % Priorytet
	\priority{10}
	% Aktorzy
	\actors{Użytkownik, Gość}
	% Przebieg główny
	\main{Ten przypadek użycia zaczyna się gdy Użytkownik chcę korzystać z Systemu Grapics.}
	\begin{enumerate}
		\item Logowanie
        \item Rejestracja
	\end{enumerate}
    \beginalternatives
	\alternative{Użytkownik nie jest zarejestrowany}{Jeśli użytkownik nie jest jeszcze zarejstrowany, nie może on się zalogować. Jeśli nie przejdzie procesu rejestracji, nie będzie on mógł pomyślnie przejśćprocesu uwierzytelniania co kończy ten przypadek użycia. }
	%\alternative{Dodatkowa pozycja}{W ten sposób można dodawać kolejne przebiegi alternatywne \ldots}
	% Diagram
	\diagram{uwierzytelnianie.pdf}{1} % drugi argument to szerokość obrazu (1.0 = szerokość strony)
	% Pre-conditions
	\precond{Brak. }
	% Post-conditions
	\postcond{Jeśli przypadek użycia został pomyślnie zrealizowany, aktor jest uewierzytelniony w Systemie.}
    % Referencje
	\references{\rf{1} \rf{2} \rf{4}}
\end{usecase}

\begin{usecase}{Logowanie}
\label{logowanie}
	% Opis przypadku użycia
	\desc{Ten przypadek użycia opisuje jak użytkownik loguje się do systemu Grapics.}
    % Priorytet
	\priority{10}
	% Aktorzy
	\actors{Użytkownik}
	% Przebieg główny
	\main{Ten przypadek użycia zaczyna się gdy użytkownik próbuje zalogować się do systemu.}
	\begin{enumerate}
		\item System prosi Użytkownika o zalogowanie
		\item Użytkownik podaje swoją nazwę użytkownika oraz hasło
		\item System potwierdza poprawność danych i loguje użytkownika do systemu
	\end{enumerate}
	% Przebiegi alternatywne
	\beginalternatives
	\alternative{Nieprawidłowa nazwa użytkownika/hasło}{Jeśli użytkownik podał złą nazwą użytkownika i/lub hasło System wyświetla odpowiedni komunikat błędu. Aktor w tym momencie może zdecydować by wrócić do początku przebiegu podstawowego lub anulowa logowanie, co kończy ten przypadek użycia}
	%\alternative{Dodatkowa pozycja}{W ten sposób można dodawać kolejne przebiegi alternatywne \ldots}
	% Diagram
	\diagram{logowanie.pdf}{1} % drugi argument to szerokość obrazu (1.0 = szerokość strony)
    % Diagram sekwencji
	\sekwencja{logowanie_sekwencja.pdf}{1}
	% Pre-conditions
	\precond{Użytkownik jest zarejestrowany w systemie.}
	% Post-conditions
	\postcond{Jeśli przypadek użycia został pomyślnie zrealizowany, aktor jest zalogowany do systemu. W przeciwnej sytuacji stan systemu pozostaje niezmieniony.}
    % Referencje
	\references{\rf{2}}
\end{usecase}

\begin{usecase}{Rejestracja}
	% Opis przypadku użycia
	\desc{Ten przypadek użycia opisuje proces zakładania konta w systemu Grapics.}
    % Priorytet
	\priority{10}
	% Aktorzy
	\actors{Administrator, Gość}
	% Przebieg główny
	\main{Ten przypadek użycia zaczyna się gdy użytkownik próbuje zarejestrować się w systemie.}
	\begin{enumerate}
		\item System prosi Użytkownika o wypełnienie formularza rejestracyjnego
		\item Użytkownik wypełnia formularz
		\item System potwierdza poprawność danych
        \item Administrator zatwierdza konto nowo-zarejestrowanego użytkownika
	\end{enumerate}
	% Przebiegi alternatywne
	\beginalternatives
	\alternative{Nieprawidłowa nazwa użytkownika/hasło}{Jeśli użytkownik podał znazwę użytkownika, która już jest używana w systemie lub/i hasło, które nie spełnia wymogów systemowych, System wyświetla odpowiedni komunikat błędu. Aktor w tym momencie może zdecydować by wrócić do początku przebiegu podstawowego lub anulować rejestrację, co kończy ten przypadek użycia}
    \alternative{Administrator nie zatwierdzi konta}{Administrator zatwierdza tylko konta, których dane pokrywają się z wewnętrzną listą klientów. Jeśli nieuprawniona osoba próbowała założyć konto, nie zostanie ono zatwierdzone co kończy ten przypadek użycia}
	%\alternative{Dodatkowa pozycja}{W ten sposób można dodawać kolejne przebiegi alternatywne \ldots}
	% Diagram
	\diagram{rejestracja.pdf}{1} % drugi argument to szerokość obrazu (1.0 = szerokość strony)
    % Diagram sekwencji
	\sekwencja{rejestracja_sekwencja.pdf}{1}
	% Pre-conditions
	\precond{Brak.}
	% Post-conditions
	\postcond{Jeśli przypadek użycia został pomyślnie zrealizowany, aktor jest zarejestrowany w Systemie i może przytąpić do logowania się. W przeciwnej sytuacji stan systemu pozostaje niezmieniony.}
    % Referencje
	\references{\rf{1} \rf{4} \rnf{1} \rnf{2}}
\end{usecase}


\section{Moduł administracyjny}
\begin{usecase}{Zarządzanie Systemem}
	% Opis przypadku użycia
	\desc{Ten ogólny przypadek użycia opisuje jak Administrator może zarządzać Systemem.}
    % Priorytet
	\priority{9}
	% Aktorzy
	\actors{Administrator}
	% Przebieg główny
	\main{Ten przypadek użycia zaczyna się gdy Administrator wejdzie do panelu Administratora}
	\begin{enumerate}
		\item Administrator robi backup danych
		\item Administrator wykonuje przywracanie systemu
		\item Administrator przegląda logi systemu
        \item Administrator przeprwowadza rekonfigurację systemu
        \item Administrator zarządza kontem Użytkownika
	\end{enumerate}
	% Przebiegi alternatywne
	\beginalternatives
	\alternative{Administrator próbował wykonać przywracanie systemu w niedozwolonym momencie}{ System nie pozwoli Administratorowi wykonać takiej opracji w innym momencie niż ten ściśle określony w kontrakcie}
	%\alternative{Dodatkowa pozycja}{W ten sposób można dodawać kolejne przebiegi alternatywne \ldots}
	% Diagram
	\diagram{zarzadzanie_systemem.pdf}{1} % drugi argument to szerokość obrazu (1.0 = szerokość strony)
	% Pre-conditions
	\precond{Administrator jest zalogowany w systemie. }
	% Post-conditions
	\postcond{Jeśli przypadek użycia został pomyślnie zrealizowany, stan danego konta użytkownika został zmieniony.}
    % Referencje
	\references{\rf{5} \rf{6} \rf{7} \rf{8} \rf{9} \rf{10} \rnf{2} \rnf{4} \rnf{5}}
\end{usecase}

\begin{usecase}{Zarządzanie kontem Użytkownika}
	% Opis przypadku użycia
	\desc{Ten ogólny przypadek użycia opisuje jak Administrator zarządza kontem użytkownika.}
    % Priorytet
	\priority{9}
	% Aktorzy
	\actors{Administrator}
	% Przebieg główny
	\main{Ten przypadek użycia zaczyna się gdy Administrator wejdzie do panelu Administratora}
	\begin{enumerate}
		\item Administrator Zatwierdza konto nowo-zarejestrowanego Użytkownika
		\item Administrator zmienia uprawnienia użytkownika
		\item Administrator usuwa konto użytkownika
	\end{enumerate}
	% Przebiegi alternatywne
	\beginalternatives
	\alternative{Administrator próbował zarządać kontem innego Administratora}{ Jeśli Administrator próbował zarządzać kontem innego Administratora, System odrzuca zmiany co kończy ten przypadek użycia}
	%\alternative{Dodatkowa pozycja}{W ten sposób można dodawać kolejne przebiegi alternatywne \ldots}
	% Diagram
	\diagram{zarzadzanie_kontem.pdf}{1} % drugi argument to szerokość obrazu (1.0 = szerokość strony)
	% Pre-conditions
	\precond{Administrator jest zalogowany w Systemie. }
	% Post-conditions
	\postcond{Jeśli przypadek użycia został pomyślnie zrealizowany, stan danego konta użytkownika został zmieniony.}
    % Referencje
	\references{\rf{5} \rf{6} \rf{7}}
\end{usecase}

\begin{usecase}{Zatwierdzanie konta}
	% Opis przypadku użycia
	\desc{Ten przypadek użycia opisuje proces zatwierdzania kont nowo-zarejestrowanych użytkowników w systemie Grapics.}
    % Priorytet
	\priority{10}
	% Aktorzy
	\actors{Administrator}
	% Przebieg główny
	\main{Ten przypadek użycia zaczyna się gdy Administrator otrzymał powiadomienie o nowo-utworzonych kontach się w systemie.}
	\begin{enumerate}
		\item System wyświetla nowych użytkowników w panelu Administratorskim
        \item Administrator wybiera jedno z kont
		\item Administrator weryfikuje i zatwierdza wybrane konto
	\end{enumerate}
	% Przebiegi alternatywne
	\beginalternatives
	\alternative{Osoba nieupowazniona próbowała założyć konto}{Jeśli osoba nieupoważniona, to znaczy taka, która nie jest klientem, próbowała założyć konto w systemie, konto zostaje odrzucone przez administratora}
	%\alternative{Dodatkowa pozycja}{W ten sposób można dodawać kolejne przebiegi alternatywne \ldots}
	% Diagram
	\diagram{zatwierdzanie_konta.pdf}{1} % drugi argument to szerokość obrazu (1.0 = szerokość strony)
    % Diagram sekwencji
	\sekwencja{zatwierdzenie_konta_sekwencja.pdf}{1}
	% Pre-conditions
	\precond{Administrator jest zalogowany w Systemie oraz istnieją niezatwierdzone konta w Systemie.}
	% Post-conditions
	\postcond{Jeśli przypadek użycia został pomyślnie zrealizowany, konto jest potwierdzone przez Administratora w Systemie i może on posłużyć do zalogowania się. W przeciwnej sytuacji stan systemu pozostaje niezmieniony.}
    % Referencje
	\references{\rf{5}}
\end{usecase}

\begin{usecase}{Usuwanie konta}
	% Opis przypadku użycia
	\desc{Ten przypadek użycia opisuje proces usuwania kont użytkowników w systemie Grapics.}
    % Priorytet
	\priority{8}
	% Aktorzy
	\actors{Administrator}
	% Przebieg główny
	\main{Ten przypadek użycia zaczyna się gdy Administrator otrzymał powiadomienie o wygaścięciu ważności konta w Systemie Grapics.}
	\begin{enumerate}
		\item System wyświetla użytkowników w panelu Administratorskim
        \item Administrator wybiera jedno z kont
		\item Administrator weryfikuje i usuwa wybrane konto
	\end{enumerate}
	% Przebiegi alternatywne
	%\alternative{Dodatkowa pozycja}{W ten sposób można dodawać kolejne przebiegi alternatywne \ldots}
	% Diagram
	\diagram{usuwanie_konta.pdf}{1} % drugi argument to szerokość obrazu (1.0 = szerokość strony)
    % Diagram sekwencji
	\sekwencja{usuwanie_konta_sekwencja.pdf}{1}
	% Pre-conditions
	\precond{Administrator jest zalogowany w Systemie oraz dostał powiadomienie o wygaśnięciu wazności konta użytkownika.}
	% Post-conditions
	\postcond{Jeśli przypadek użycia został pomyślnie zrealizowany, konto jest usunięte przez Administratora w Systemie i nie może ono posłużyć do zalogowania się. W przeciwnej sytuacji stan systemu pozostaje niezmieniony.}
    % Referencje
	\references{\rf{7}}
\end{usecase}

\begin{usecase}{Zmiana uprawnień}
	% Opis przypadku użycia
	\desc{Ten przypadek użycia opisuje proces zmieniania uprawnież kont użytkowników w systemie Grapics.}
    % Priorytet
	\priority{8}
	% Aktorzy
	\actors{Administrator}
	% Przebieg główny
	\main{Ten przypadek użycia zaczyna się gdy Administrator otrzymał powiadomienie o potrzebie zmiany uprawnień konta w Systemie Grapics.}
	\begin{enumerate}
		\item System wyświetla użytkowników w panelu Administratorskim
        \item Administrator wybiera jedno z kont
		\item Administrator weryfikuje i zmienia uprawnienia wybranengo konta
	\end{enumerate}
	% Przebiegi alternatywne
	%\alternative{Dodatkowa pozycja}{W ten sposób można dodawać kolejne przebiegi alternatywne \ldots}
	% Diagram
	\diagram{zmiana_uprawnien.pdf}{1} % drugi argument to szerokość obrazu (1.0 = szerokość strony)
    % Diagram sekwencji
	\sekwencja{zmiana_uprawnien_sekwencja.pdf}{1}
	% Pre-conditions
	\precond{Administrator jest zalogowany w Systemie oraz dostał powiadomienie o potrzebie zmiany uprawnień danego konta użytkownika.}
	% Post-conditions
	\postcond{Jeśli przypadek użycia został pomyślnie zrealizowany, uprawnienia danego konta zostały zmienione. W przeciwnej sytuacji stan systemu pozostaje niezmieniony.}
    % Referencje
	\references{\rf{6}}
\end{usecase}

\begin{usecase}{Backup danych}
	% Opis przypadku użycia
	\desc{Ten przypadek użycia opisuje proces robienia backupu danych przez Administratora.}
    % Priorytet
	\priority{10}
	% Aktorzy
	\actors{Administrator}
	% Przebieg główny
	\main{Ten przypadek użycia zaczyna się gdy Administrator otrzymał powiadomienie o potrzebie zrobienia backupu danych w Systemie Grapics.}
	\begin{enumerate}
		\item Administrator wybiera opcję "Backup danych" w panelu Administratorskim
        \item Administrator zaznacza odpowiednie opcje
		\item Administrator czeka aż proces dobiegnie końca
	\end{enumerate}
	% Przebiegi alternatywne
	%\alternative{Dodatkowa pozycja}{W ten sposób można dodawać kolejne przebiegi alternatywne \ldots}
	% Diagram
	\diagram{backup_danych.pdf}{1} % drugi argument to szerokość obrazu (1.0 = szerokość strony)
    % Diagram sekwencji
	\sekwencja{backup_sekwencja.pdf}{1}
	% Pre-conditions
	\precond{Administrator jest zalogowany w Systemie oraz dostał powiadomienie o potrzebie wykonania Backupu.}
	% Post-conditions
	\postcond{Jeśli przypadek użycia został pomyślnie zrealizowany, backup Systemu został wykonany. W przeciwnej sytuacji stan systemu pozostaje niezmieniony.}
    % Referencje
	\references{\rf{8} \rnf{2}}
\end{usecase}

\begin{usecase}{Przywracanie Systemu}
	% Opis przypadku użycia
	\desc{Ten przypadek użycia opisuje proces przywracania Systemu przez Administratora.}
    % Priorytet
	\priority{10}
	% Aktorzy
	\actors{Administrator}
	% Przebieg główny
	\main{Ten przypadek użycia zaczyna się gdy Administrator otrzymał powiadomienie o potrzebie przywrócenia Systemu Grapics.}
	\begin{enumerate}
		\item Administrator odnajduje odpowiednią wersję Systemu, którą należy przywrócić
        \item Administrator wybiera opcję "Przywracanie Systemu" z panelu Administratora
        \item Administrator zaznacza odpowiednie opcje
		\item Administrator czeka aż proces dobiegnie końca
	\end{enumerate}
	% Przebiegi alternatywne
    \beginalternatives
	\alternative{Administrator próbował wykonać przywracanie systemu w niedozwolonym momencie}{ System nie pozwoli Administratorowi wykonać takiej opracji w innym momencie niż ten ściśle określony w kontrakcie}
	%\alternative{Dodatkowa pozycja}{W ten sposób można dodawać kolejne przebiegi alternatywne \ldots}
	% Diagram
	\diagram{przywracanie_systemu.pdf}{1} % drugi argument to szerokość obrazu (1.0 = szerokość strony)
    % Diagram sekwencji
	\sekwencja{przywracanie_systemu_sekwencja.pdf}{1}
	% Pre-conditions
	\precond{Administrator jest zalogowany w Systemie oraz dostał powiadomienie o potrzebie przywrócenia Systemu.}
	% Post-conditions
	\postcond{Jeśli przypadek użycia został pomyślnie zrealizowany, odpowiednia wersja Systemu została przywrócona. W przeciwnej sytuacji stan systemu pozostaje niezmieniony.}
    % Referencje
	\references{\rf{10}}
\end{usecase}

\begin{usecase}{Przeglądanie logów Systemu}
	% Opis przypadku użycia
	\desc{Ten przypadek użycia opisuje przeglądania logów systemu przez Administratora.}
    % Priorytet
	\priority{7}
	% Aktorzy
	\actors{Administrator}
	% Przebieg główny
	\main{Ten przypadek użycia zaczyna się gdy Administrator zamierza przejrzeć logi w Systemie Grapics.}
	\begin{enumerate}
		\item Administrator wybiera opcję "Logi" w panelu Administratora
        \item Administrator zaznacza odpowiednie opcje
		\item Administrator czeka aż zbieranie logów dobiegnie końca
        \item Administrator otrzymuje logi w formacie .txt
	\end{enumerate}
	% Przebiegi alternatywne
	%\alternative{Dodatkowa pozycja}{W ten sposób można dodawać kolejne przebiegi alternatywne \ldots}
	% Diagram
	\diagram{przegladanie_logow.pdf}{1} % drugi argument to szerokość obrazu (1.0 = szerokość strony)
    % Diagram sekwencji
	\sekwencja{przegladanie_logow_sekwencja.pdf}{1}
	% Pre-conditions
	\precond{Administrator jest zalogowany w Systemie.}
	% Post-conditions
	\postcond{Jeśli przypadek użycia został pomyślnie zrealizowany, logi Systemu zostały wygenerowane do pliku .txt. W przeciwnej sytuacji stan systemu pozostaje niezmieniony.}
    % Referencje
	\references{\rf{9}}
\end{usecase}

\begin{usecase}{Rekonfiguracja Systemu}
	% Opis przypadku użycia
	\desc{Ten przypadek użycia opisuje proces rekonfiguracji Systemu przez Administratora.}
    % Priorytet
	\priority{9}
	% Aktorzy
	\actors{Administrator}
	% Przebieg główny
	\main{Ten przypadek użycia zaczyna się gdy Administrator otrzymał powiadomienie o potrzebie rekonfiguracji Systemu Grapics.}
	\begin{enumerate}
		\item Administrator wybiera opcję "Rekonfiguracja Systemu" w panelu Administratorskim
        \item Administrator zaznacza odpowiednie opcje
		\item Administrator czeka aż proces dobiegnie końca
	\end{enumerate}
	% Przebiegi alternatywne
    \beginalternatives
	\alternative{Administrator próbował wykonać Rekonfiguracje Systemu w niedozwolonym momencie}{ System nie pozwoli Administratorowi wykonać takiej opracji w innym momencie niż ten ściśle określony w kontrakcie}
	%\alternative{Dodatkowa pozycja}{W ten sposób można dodawać kolejne przebiegi alternatywne \ldots}
	% Diagram
	\diagram{rekonfiguracja_systemu.pdf}{1} % drugi argument to szerokość obrazu (1.0 = szerokość strony)
    % Diagram sekwencji
	\sekwencja{rekonfiguracja_sekwencja.pdf}{1}
	% Pre-conditions
	\precond{Administrator jest zalogowany w Systemie oraz dostał powiadomienie o potrzebie rekonfiguracji Systemu.}
	% Post-conditions
	\postcond{Jeśli przypadek użycia został pomyślnie zrealizowany, rekonfiguracja Systemu została wykonana. W przeciwnej sytuacji stan systemu pozostaje niezmieniony.}
    % Referencje
	\references{\rf{10}}
\end{usecase}
\section{Moduł główny}
\begin{usecase}{Use Case name}
	% Opis przypadku użycia
	\desc{Use Case description}
	% Priorytet
	\priority{3}
	% Aktorzy
	\actors{Aktor1, Aktor2, Aktor3}
	% Przebieg główny
	\main{Main flow description}
	\begin{enumerate}
		\item Action 1
		\item Action 2
		\item Action 3
	\end{enumerate}
	% Przebiegi alternatywne
	\beginalternatives
	\alternative{Alternative flow 1}{Actions \ldots}
	\alternative{Alternative flow 2}{Actions \ldots}
	\alternative{Alternative flow 3}{Actions \ldots}
	% Diagram
	\diagram{todo.pdf}{0.6} % drugi argument to szerokość obrazu (1.0 = szerokość strony)
	% Diagram sekwencji
	\sekwencja{todo.pdf}{0.6}  % drugi argument to szerokość obrazu (1.0 = szerokość strony)
    % Diagram sekwencji
	\klasa{todo.pdf}{0.6}  % drugi argument to szerokość obrazu (1.0 = szerokość strony)
	% Pre-conditions
	\precond{\todo}
	% Post-conditions
	\postcond{\todo}
	% Referencje
	\references{\rf{1} \rf{2} \rnf{1} \rnf{2}}
\end{usecase}
% może nie usuwajcie tego szablonu z góry, będzie można sobie kopiować


\begin{usecase}{Przeglądanie Projektów}
\label{przegladanie_projektow}
	% Opis przypadku użycia
	\desc{Użytkownik przegląda Projekty, w których funkcjonuje w roli Członka Projektu}
	% Priorytet
	\priority{10}
	% Aktorzy
	\actors{Członek Projektu}
	% Przebieg główny
	\begin{enumerate}
      \item Użytkownik wchodzi w sekcję "moje projekty"
      \item Użytkownik zaznacza opcję filtrowania projektów, w których jest członkiem
      \item System wyświetla listę projektów, w których Użytkownik funkcjonuje w roli Członka Projektu.
	\end{enumerate}
	% Diagram
	\diagram{przegladanie_projektow.pdf}{0.9} % drugi argument to szerokość obrazu (1.0 = szerokość strony)
	% Pre-conditions
	% Diagram sekwencji
	\parindent=0cm
	\textbf{Diagram sekwencji}
	\ref{Przegladanie projektow}
	% Diagram klas
	%\klasa{przegladanie_logow_klas.pdf}{1}
	\precond{Użytkownik jest zalogowany (\ref{logowanie})}
	\references{\rf{13}}
\end{usecase}


\begin{usecase}{Tworzenie Projektu}
\label{tworzenie_projektu}
	% Opis przypadku użycia
	\desc{Użytkownik tworzy nowy Projekt}
	% Priorytet
	\priority{10}
	% Aktorzy
	\actors{Członek Projektu, Twórca Projektu}
	% Przebieg główny
	\begin{enumerate}
    \item Użytkownik przegląda istniejące Projekty (\ref{przegladanie_projektow})
    \item Użytkownik wybiera opcję  “Utwórz nowy projekt”
    \item Użytkownik wypełnia formularz
    \item Użytkownik zatwierdza utworzenie nowego projektu.
	\end{enumerate}
	% Diagram
	\diagram{tworzenie_projektu.pdf}{0.9} % drugi argument to szerokość obrazu (1.0 = szerokość strony)
	% Pre-conditions
	\precond{Użytkownik jest zalogowany}
	% Post-conditions
	\postcond{Został utworzony nowy Projekt, a Użytkownik, który go stworzył, zostaje Twórcą tego Projektu}
	\references{\rf{11}}
\end{usecase}


\begin{usecase}{Edycja Projektu}
	% Opis przypadku użycia
	\desc{Twórca Projektu edytuje swój Projekt}
	% Priorytet
	\priority{10}
	% Aktorzy
	\actors{Członek Projektu, Twórca Projektu}
	% Przebieg główny
	\begin{enumerate}
    \item Twórca Projektu przegląda istniejące Projekty (\ref{przegladanie_projektow})
    \item Twórca Projektu wybiera Projekt
    \item Twórca Projektu dokonuje zmian w formularzu
    \item Twórca Projektu zatwierdza zmiany
	\end{enumerate}
	% Diagram
	\diagram{edycja_projektu.pdf}{0.9} % drugi argument to szerokość obrazu (1.0 = szerokość strony)
	% Diagram sekwencji
	%\sekwencja{edycja_projektu_sekwencja_fit.pdf}{0.9}
	\parindent=0cm
	\textbf{Diagram sekwencji}
	\ref{Edycja projektu}
	% Pre-conditions
	\precond{Użytkownik jest zalogowany (\ref{logowanie}) i jest Twórcą edytowanego Projektu (\ref{tworzenie_projektu})}
	% Post-conditions
	\postcond{Właściwości Projektu zostały zmienione}
	\references{\rf{19}}
\end{usecase}


\begin{usecase}{Zamknięcie projektu}
	% Opis przypadku użycia
	\desc{Twórca Projektu zgłasza do zamknięcia swój Projekt}
	% Priorytet
	\priority{10}
	% Aktorzy
	\actors{Twórca Projektu}
	% Przebieg główny
	\begin{enumerate}
	\item Twórca Projektu przegląda istniejące Projekty (\ref{przegladanie_projektow})
    \item Twórca Projektu wybiera Projekt
    \item Twórca Projektu wybiera Zgłoszenie
    \item Twórca Projektu wydaje polecenie zamknięcia projektu
	\end{enumerate}
	% Diagram
	\diagram{zamkniecie_projektu.pdf}{0.9} % drugi argument to szerokość obrazu (1.0 = szerokość strony)
	% Diagram sekwencji
	%\sekwencja{zamkniecie_projektu_sekwencja_fit.pdf}{0.9}
	\parindent=0cm
	\textbf{Diagram sekwencji}
	\ref{Zamkniecie projektu}
	% Pre-conditions
	\precond{Użytkownik jest zalogowany (\ref{logowanie}) i jest Twórcą zamykanego Projektu (\ref{tworzenie_projektu})}
	% Post-conditions
	\postcond{Projekt został zamknięty, Użytkownik nie jest już Twórcą Projektu}
	\references{\rf{20}}
\end{usecase}


\begin{usecase}{Tworzenie Zgłoszenia}
\label{tworzenie_zgloszenia}
	% Opis przypadku użycia
	\desc{Opisuje proces, w którym Użytkownik będący Członkiem Projektu dodaje nowe Zgłoszenie do Projektu}
	% Priorytet
	\priority{10}
	% Aktorzy
	\actors{Członek Projektu}
	% Przebieg główny
	\begin{enumerate}
      \item Członek Projektu przegląda istniejące Projekty (\ref{przegladanie_projektow})
      \item Członek Projektu wybiera odpowiedni Projekt
      \item Członek Projektu uzupełnia szczegóły nowego Zgłoszenia
      \item Członek Projektu zatwierdza powstanie nowego Zgłoszenia
	\end{enumerate}
	% Przebiegi alternatywne
	\beginalternatives
	\alternative{Wybranie Osoby Odpowiedzialnej za Zgłoszenie}{Członek Projektu wybiera Osobę Odpowiedzialną za utworzone zgłoszenie spośród Członków Projektu (\ref{wyznaczenie_osoby_odpowiedzialnej})}
	% Diagram
	\diagram{tworzenie_zgloszenia.pdf}{0.9} % drugi argument to szerokość obrazu (1.0 = szerokość strony)
	% Diagram sekwencji
	%\sekwencja{tworzenie_zgloszenia_sekwencja_fit.pdf}{0.9}
	\parindent=0cm
	\textbf{Diagram sekwencji}
	\ref{Tworzenie zgloszenia}
	% Pre-conditions
	\precond{Użytkownik (Członek Projektu) jest zalogowany (\ref{logowanie}) i jest członkiem przynajmniej jednego Projektu}
	% Post-conditions
	\postcond{Powstało nowe Zgłoszenie w Projekcie, Członek Projektu staje się Twórca Zgłoszenia dla utworzonego Zgłoszenia}
	\references{\rf{14}}
\end{usecase}


\begin{usecase}{Zamknięcie Zgłoszenia}
	% Opis przypadku użycia
	\desc{Twórca Zgłoszenia zamyka swoje Zgłoszenie}
	% Priorytet
	\priority{10}
	% Aktorzy
	\actors{Twórca Zgłoszenia}
	\begin{enumerate}
    \item Twórca Zgłoszenia przegląda istniejące Projekty (\ref{przegladanie_projektow})
    \item Twórca Zgłoszenia wybiera Projekt
    \item Twórca Zgłoszenia wybiera Zgłoszenie
    \item Twórca Zgłoszenia wydaje polecenie zamknięcia zgłoszenia
    \end{enumerate}
	% Diagram
	\diagram{zamkniecie_zgloszenia.pdf}{0.9} % drugi argument to szerokość obrazu (1.0 = szerokość strony)
	% Diagram sekwencji
	%\sekwencja{zamkniecie_zgloszenia_sekwencja_fit.pdf}{0.9}
	\parindent=0cm
	\textbf{Diagram sekwencji}
	\ref{Zamkniecie zgloszenia}
	% Pre-conditions
    \precond{Użytkownik jest zalogowany (\ref{logowanie}) i jest Twórcą zamykanego Zgłoszenia (\ref{tworzenie_zgloszenia})}
	% Post-conditions
	\postcond{Zgłoszenie zostało zamknięte, Użytkownik nie jest już Twórcą Zgłoszenia}
	\references{\rf{21}}
\end{usecase}


\begin{usecase}{Zmiana widoczności Zgłoszenia}
	% Opis przypadku użycia
	\desc{Twórca Zgłoszenia zmienia widoczność swojego Zgłoszenia}
	% Priorytet
	\priority{5}
	% Aktorzy
	\actors{Twórca Zgłoszenia}
    % Przebieg główny
	\begin{enumerate}
    \item Twórca Zgłoszenia przegląda istniejące Projekty (\ref{przegladanie_projektow})
    \item Twórca Zgłoszenia wybiera Projekt
    \item Twórca Zgłoszenia wybiera Zgłoszenie
    \item Twórca Zgłoszenia wydaje polecenia ukrycia/odkrycia Zgłoszenia
	\end{enumerate}
	% Diagram
	\diagram{zmiana_widocznosci_zgloszenia.pdf}{0.9} % drugi argument to szerokość obrazu (1.0 = szerokość strony)
	% Diagram sekwencji
	%\sekwencja{zmiana_widocznosci_zgloszenia_sekwencja_fit.pdf}{0.9}
	\parindent=0cm
	\textbf{Diagram sekwencji}
	\ref{Zmiana widocznosci zgloszenia}
	% Pre-conditions
	\precond{Użytkownik jest zalogowany (\ref{logowanie}) i jest Twórcą edytowanego Zgłoszenia (\ref{tworzenie_zgloszenia})}
	% Post-conditions
	\postcond{Widoczność Zgłoszenia została zmieniona}
		\references{\rf{22}}
\end{usecase}


\begin{usecase}{Edycja Zgłoszenia}
\label{edycja_zgloszenia}
	% Opis przypadku użycia
	\desc{Twórca Zgłoszenia edytuje podstawowe informacje o Zgłoszeniu}
	% Priorytet
	\priority{9}
	% Aktorzy
	\actors{Twórca Zgłoszenia}
    % Przebieg główny
	\begin{enumerate}
	\item Twórca Zgłoszenia przegląda istniejące Projekty (\ref{przegladanie_projektow})
    \item Twórca Zgłoszenia wybiera Projekt
    \item Twórca Zgłoszenia wybiera Zgłoszenie
    \item Twórca Zgłoszenia wydaje polecenie edycji Zgłoszenia
    \item Twórca Zgłoszenia edytuje formularz
    \item Twórca Zgłoszenia zatwierdza zmiany
	\end{enumerate}
	% Diagram
	\diagram{edycja_zgloszenia.pdf}{0.9} % drugi argument to szerokość obrazu (1.0 = szerokość strony)
	% Diagram sekwencji
	%\sekwencja{edycja_zgloszenia_sekwencja_fit.pdf}{0.9}
	\parindent=0cm
	\textbf{Diagram sekwencji}
	\ref{Edycja zgloszenia}
	% Pre-conditions
	\precond{Użytkownik jest zalogowany (\ref{logowanie}), należy do przynajmniej jednego Projektu i jest Twórcą przynajmniej jednego Zgłoszenia (\ref{tworzenie_zgloszenia})}
	% Post-conditions
	\postcond{Właściwości Zgłoszenia zostały zmienione}
	\references{\rf{23}}
\end{usecase}


\begin{usecase}{Wyznaczenie Osoby Odpowiedzialnej}
\label{wyznaczenie_osoby_odpowiedzialnej}
	% Opis przypadku użycia
	\desc{Twórca Zgłoszenia wyznacza Użytkownika jako Odbiorcę Zgłoszenia dla danego Zgłoszenia}
	% Priorytet
	\priority{9}
	% Aktorzy
	\actors{Twórca Zgłoszenia, Osoba Odpowiedzialna, Użytkownik}
	% Przebieg główny
	\begin{enumerate}
    \item Twórca Zgłoszenia przegląda istniejące Projekty (\ref{przegladanie_projektow})
  	\item Twórca Zgłoszenia wybiera Projekt
  	\item Twórca Zgłoszenia wybiera Zgłoszenie
  	\item Twórca Zgłoszenia wydaje polecenie edycji Zgłoszenia (\ref{edycja_zgloszenia})
  	\item Twórca Zgłoszenia wybiera z listy Użytkowników Osobę Odpowiedzialną
  	\item Twórca Zgłoszenia zatwierdza zmiany
	\end{enumerate}
	% Diagram
	\diagram{wyznaczenie_osoby_odpowiedzialnej.pdf}{0.9} % drugi argument to szerokość obrazu (1.0 = szerokość strony)
	% Pre-conditions
	\precond{Użytkownik jest zalogowany (\ref{logowanie}), należy do przynajmniej jednego Projektu i jest Twórcą przynajmniej jednego Zgłoszenia (\ref{tworzenie_zgloszenia})}
	% Post-conditions
	\postcond{Do zgłoszenia została przypisana osoba odpowiedzialna}
	\references{\rf{24}}
\end{usecase}


\begin{usecase}{Edycja Grafu}
	% Opis przypadku użycia
	\desc{Twórca Projektu planuje kolejne etapy Projektu w formie Zgłoszeń, przechowywanych w postaci Grafu}
	% Priorytet
	\priority{10}
	% Aktorzy
	\actors{Twórca Projektu}
	% Przebieg główny
	\begin{enumerate}
    \item Twórca Projektu przegląda istniejące Projekty (\ref{przegladanie_projektow})
    \item Twórca Projektu wybiera Projekt
    \item Twórca Projektu wybiera istniejące Zgłoszenie
    \item Twórca Projektu wyznacza rodzica Zgłoszenia lub usuwa istniejące dzieci
	\end{enumerate}
	% Przebiegi alternatywne
	% Diagram
	\diagram{tworzenie_grafu.pdf}{0.9} % drugi argument to szerokość obrazu (1.0 = szerokość strony)
	% Pre-conditions
	\precond{Użytkownik jest zalogowany (\ref{logowanie})  i jest Twórcą przynajmniej jednego Projektu (\ref{tworzenie_projektu})}
	% Post-conditions
	\postcond{Struktura Grafu została zmieniona}
	\references{\rf{27}}
\end{usecase}


\begin{usecase}{Ustawienie szczegółowych uprawnień}
	% Opis przypadku użycia
	\desc{Twórca Projektu ustawia szczegółowe uprawnienia innym Członkom Projektu}
	% Priorytet
	\priority{6}
	% Aktorzy
	\actors{Użytkownik}
	% Przebieg główny
	\begin{enumerate}
    \item Użytkownik wchodzi w sekcję ‘swoje projekty’
    \item Użytkownik wybiera odpowiedni projekt
    \item Użytkownik wybiera opcję ‘zarządzaj Członkami Projektu’
    \item Użytkownik wybiera odpowiedniego Członka Projektu
    \item Użytkownik zmienia uprawnienia wybranego Członka Projektu
	\end{enumerate}
	% Przebiegi alternatywne
	\beginalternatives
	\alternative{Alternative flow 1}{Działanie się nie wykona, jeżeli użytkownik nie będzie miał uprawnień do nadawania uprawnień innym aktorom.}
	% Diagram
	\diagram{ustawienia_szczegolowych_uprawnien.pdf}{1.0} % drugi argument to szerokość obrazu (1.0 = szerokość strony)
	% Diagram sekwencji
	%\sekwencja{todo.pdf}{0.6}  % drugi argument to szerokość obrazu (1.0 = szerokość strony)
	% Diagram sekwencji
	%\klasa{todo.pdf}{0.6}  % drugi argument to szerokość obrazu (1.0 = szerokość strony)
	% Pre-conditions
	\precond{Użytkownik jest zalogowany i jest Twórcą przynajmniej jednego Projektu }
	% Post-conditions
	\postcond{Uprawnienia co najmniej jednego użytkownika zostały zmienione}
	% Referencje
	\references{\rf{1} \rf{19}}
\end{usecase}

\begin{usecase}{Tworzenie projektu z wieloma zgłoszeniami}
	% Opis przypadku użycia
	\desc{Użytkownik tworzy projekty zawierające wiele zgłoszeń}
	% Priorytet
	\priority{4}
	% Aktorzy
	\actors{Użytkownik}

	% Przebieg główny
	\begin{enumerate}
    \item Użytkownik wybiera opcję ‘utwórz nowy Projekt’
    \item Użytkownik wypełnia odpowiednia informacje o Nowym Projekcie
    \item Użytkownik dodaje innych użytkowników jako Członków Projektu
    \item Użytkownik tworzy Zgłoszenia w danym projekcie.
	\end{enumerate}
	% Przebiegi alternatywne
	\beginalternatives
	\alternative{Alternative flow 1}{Przy odpowiednio dużej ilości zgłoszeń system nie pozwoli tworzenie projektu}
	\alternative{Alternative flow 2}{Jeżeli użytkownik poda niepoprawne hasło system zwróci komunikat o tym}
	% Diagram
	\diagram{tworzenie_projektu_z_wieloma_zgloszeniami.pdf}{1.0} % drugi argument to szerokość obrazu (1.0 = szerokość strony)
	% Diagram sekwencji
	%\sekwencja{todo.pdf}{0.6}  % drugi argument to szerokość obrazu (1.0 = szerokość strony)
	% Diagram sekwencji
	%\klasa{todo.pdf}{0.6}  % drugi argument to szerokość obrazu (1.0 = szerokość strony)
	% Pre-conditions
	\precond{Jest Twórcą przynajmniej jednego Projektu}
	% Post-conditions
	\postcond{Powstanie co najmniej jednego zgłoszenia}
	% Referencje
	\references{\rf{11} \rf{14}}
\end{usecase}

\begin{usecase}{Udzielenie informacji zwrotnej}
	% Opis przypadku użycia
	\desc{Odbiorca Zgłoszenia udziela informacji zwrotnej Osobie Odpowiedzialnej przez zmianę statusu Zgłoszenia.}
	% Priorytet
	\priority{3}
	% Aktorzy
	\actors{Użytkownik}
	% Przebieg główny
	\begin{enumerate}
    \item Odbiorca Zgłoszenia  loguje się
    \item Odbiorca Zgłoszenia wchodzi w sekcję "moje projekty"
    \item Odbiorca Zgłoszenia wybiera Projekt
    \item Odbiorca Zgłoszenia wybiera Zgłoszenie
    \item Odbiorca Zgłoszenia edytuje Zgłoszenie
    \item Odbiorca Zgłoszenia ustawia odpowiedni status Zgłoszenia i zamieszcza komentarz
    \item Odbiorca Zgłoszenia zatwierdza zmiany
	\end{enumerate}
	% Przebiegi alternatywne
	\beginalternatives
	\alternative{Alternative flow 1}{Działanie się nie powiedzie, jeżeli użtkownik będzie miał status projektu zamkniętego}
	% Diagram
	\diagram{udzial_informacji_zwrotnej.pdf}{1.0} % drugi argument to szerokość obrazu (1.0 = szerokość strony)
	% Diagram sekwencji
	%\sekwencja{todo.pdf}{0.6}  % drugi argument to szerokość obrazu (1.0 = szerokość strony)
	% Diagram sekwencji
	%\klasa{todo.pdf}{0.6}  % drugi argument to szerokość obrazu (1.0 = szerokość strony)
	% Pre-conditions
	\precond{Odbiorca Zgłoszenia jest twórcą przynajmniej jednego Projektu}
	% Post-conditions
	\postcond{Zmiana statusu}
	% Referencje
	\references{\rf{11} \rf{13} \rf{18} \rf{20}}
\end{usecase}

\begin{usecase}{Aktualizacja zasobów}
	% Opis przypadku użycia
	\desc{Odbiorca Zgłoszenia aktualizuje Zasoby Zgłoszenia.}
	% Priorytet
	\priority{8}
	% Aktorzy
	\actors{Użytkownik}
	% Przebieg główny
	\begin{enumerate}
    \item Odbiorca Zgłoszenia wchodzi na stronę
    \item Ta osoba loguje się jako root
    \item Próbuje zmienić “propozycję”
    \item Jeżeli taka jest to wraca do punktu 3
    \item Zmiana się powiodła
	\end{enumerate}
	% Przebiegi alternatywne
	\beginalternatives
	\alternative{Alternative flow 1}{Jeżeli po zalogowaniu nie będzie "propozycji" to działanie się nie powiedzie}
	% Diagram
	\diagram{aktualizacja_zasobow.pdf}{1.0} % drugi argument to szerokość obrazu (1.0 = szerokość strony)
	% Diagram sekwencji
	%\sekwencja{todo.pdf}{0.6}  % drugi argument to szerokość obrazu (1.0 = szerokość strony)
	% Diagram sekwencji
	%\klasa{todo.pdf}{0.6}  % drugi argument to szerokość obrazu (1.0 = szerokość strony)
	% Pre-conditions
	\precond{Odbiorca Zgłoszenia jest twórcą przynajmniej jednego Projektu}
	% Post-conditions
	\postcond{Zmiana zasobów zgłoszenia}
	% Referencje
	\references{\rf{17} \rf{18} \rf{19}}
\end{usecase}

\begin{usecase}{Pobieranie załączników}
	% Opis przypadku użycia
	\desc{Twórca/Odbiorca Zgłoszenia/Odbiorca pobiera załączniki dodane do Zgłoszeń.}
	% Priorytet
	\priority{2}
	% Aktorzy
	\actors{Użytkownik, Twórca projektu}
	% Przebieg główny
	\begin{enumerate}
    \item Osoba wchodzi na stronę
    \item Loguje się podając Imię Nazwisko i nr. PESEL
    \item Osoba wyszukuje zgłoszenie
    \item Pobiera załącznik.
    \item Jeżeli to jest Twórca projektu, jest opcja pobrać wszystkie załączniki
	\end{enumerate}
	% Przebiegi alternatywne
	% Diagram
	\diagram{pobieranie_zalacznikow.pdf}{1.0} % drugi argument to szerokość obrazu (1.0 = szerokość strony)
	% Diagram sekwencji
	%\sekwencja{todo.pdf}{0.6}  % drugi argument to szerokość obrazu (1.0 = szerokość strony)
	% Diagram sekwencji
	%\klasa{todo.pdf}{0.6}  % drugi argument to szerokość obrazu (1.0 = szerokość strony)
	% Pre-conditions
	\precond{Odbiorca Zgłoszenia jest twórcą przynajmniej jednego Projektu (\ref{tworzenie_projektu})}
	% Post-conditions
	\postcond{Zmiana zasobów zgłoszenia}
	% Referencje
	\references{\rf{31}}
\end{usecase}

\begin{usecase}{Przeszukiwanie zgłoszeń}
	% Opis przypadku użycia
	\desc{Członek Projektu przeszukuje Zgłoszenia w Projekcie.}
	% Priorytet
	\priority{7}
	% Aktorzy
	\actors{Użytkownik}
	% Przebieg główny
	\begin{enumerate}
    \item Użytkownik loguje się
    \item W polu “Znajdź zgłoszenie” wpisuje kluczowe słowa
    \item Przeglądarka wyświetla znalezione zgłoszenia
    \item Jeżeli nie znalazł nic wyświetla komunikat “Takich zgłoszeń nie ma”
	\end{enumerate}
	% Przebiegi alternatywne
	\beginalternatives
	\alternative{Alternative flow 1}{Jeżeli nie będzie znaleziono żadnego ze zgłoszeń aplikacja wyświetli odpowiedni komunikat}
	% Diagram
	\diagram{przeszukiwanie_zgloszen.pdf}{1.0} % drugi argument to szerokość obrazu (1.0 = szerokość strony)
	%\sekwencja{todo.pdf}{0.6}  % drugi argument to szerokość obrazu (1.0 = szerokość strony)
	% Diagram sekwencji
	%\klasa{todo.pdf}{0.6}  % drugi argument to szerokość obrazu (1.0 = szerokość strony)
	% Pre-conditions
	\precond{Odbiorca Zgłoszenia jest twórcą przynajmniej jednego Projektu (\ref{tworzenie_projektu})}
	% Post-conditions
	\postcond{Uzyskanie odpowiednich zgłoszeń lub informacji że ich brak}
	% Referencje
	\references{\rf{32}}
\end{usecase}

\begin{usecase}{Wysyłanie informacji o zamknięciu projektu, "Odbiorca-Twórca"}
	% Opis przypadku użycia
	\desc{Odbiorca Zgłoszenia wysyłam Twórcy Zgłoszenia informacje, że zgłoszenie jest do zamknięcia}
	% Priorytet
	\priority{5}
	% Aktorzy
	\actors{Użytkownik, Twórca projektu}
	% Przebieg główny
	\begin{enumerate}
    \item Odbiorca Zgłoszenia loguje się do systemu
    \item Odbiorca Zgłoszenia wchodzi do sekcji zgłoszenia
    \item W tym zgłoszenie jest podany adres @mail
    \item Odbiorca Zgłoszenia wysyła mail
	\end{enumerate}
	% Przebiegi alternatywne
	\beginalternatives
	\alternative{Alternative flow 1}{Jeżeli adres @mail nie będzie podany,
	system wyświetli odpowiedni komunikat}
	% Diagram
	\diagram{odbiorca_tworca.pdf}{1.0} % drugi argument to szerokość obrazu (1.0 = szerokość strony)
	%\sekwencja{todo.pdf}{0.6}  % drugi argument to szerokość obrazu (1.0 = szerokość strony)
	% Diagram sekwencji
	%\klasa{todo.pdf}{0.6}  % drugi argument to szerokość obrazu (1.0 = szerokość strony)
	% Pre-conditions
	\precond{Odbiorca i twórca muszą należeć do co najmniej jednego projektu.}
	% Post-conditions
	\postcond{Uzyskanie odpowiednich zgłoszeń lub informacji, że jest ich brak}
	% Referencje
	\references{\rf{20} \rf{21}}
\end{usecase}

\begin{usecase}{Wysłanie informacji o zamknięciu projektu, "Użytkownik-Twórca"}
	% Opis przypadku użycia
	\desc{Członek Projektu wysyła Twórcy Projektu informacje, że  projekt jest do zamknięcia.}
	% Priorytet
	\priority{5}
	% Aktorzy
	\actors{Użytkownik, Twórca projektu}

	% Przebieg główny
	\begin{enumerate}
    \item Członek projektu wchodzi na stronę
    \item Podaje Imię, Nazwisko i identyfikator
    \item Naciska na przycisk “Projekt jest do zamknięcia”
    \item Wiadomość przychodzi do Twórcy projektu
	\end{enumerate}
	% Przebiegi alternatywne
	\beginalternatives
	\alternative{Alternative flow 1}{Jeżeli adres @mail nie będzie podany,
		system wyświetli odpowiedni komunikat}
	% Diagram
	\diagram{uzytkownik_tworca.pdf}{1.0} % drugi argument to szerokość obrazu (1.0 = szerokość strony)
	%\sekwencja{todo.pdf}{0.6}  % drugi argument to szerokość obrazu (1.0 = szerokość strony)
	% Diagram sekwencji
	%\klasa{todo.pdf}{0.6}  % drugi argument to szerokość obrazu (1.0 = szerokość strony)
	% Pre-conditions
	\precond{Odbiorca i twórca muszą należeć do co najmniej jednego projektu.}
	% Post-conditions
	\postcond{Uzyskanie odpowiednich zgłoszeń lub informacji że jest ich brak}
	\references{\rf{20} \rf{21}}
\end{usecase}

\chapter{Wymagania funkcjonalne}
W niniejszym rozdziale przedstawione zostały wymagania funkcjonalne, które zdefiniowano dla poszczególnych modułów systemu. Priorytet oraz złożoność realizacji każdego z wymagań wyrażone zostały przy użyciu dziesięciostopniowej skali liniowej (1--10). Liczba 10 odpowiada najwyższemu priorytetowi i największej złożoności, natomiast liczba 1 - wartościom najniższym.
\section{Moduł uwierzytelniania}
% Moduł uwierzytelniania

\funcreq{System powinien pozwolić gościom zarejestrować się.}{10}{4}
\funcreq{System powinien pozwolić użytkownikom zalogować i wylogować się.}{10}{5}
\funcreq{System powinien pozwolić użytkownikom zmieniać swoje dane (takie jak adres e-mail, hasło).}{7}{4}
\funcreq{System powinien weryfikować dane wprowadzane przez użytkowników (takie	jak typ danych lub czy użytkownik o danej nazwie / adresie e-mail już nie istnieje).}{10}{5}
\section{Moduł administracyjny}
% Moduł administracyjny

\funcreq{System powinien umożliwiać Administratorom weryfikację nowo-zarejestrowanych użytkowników.}{10}{6}
\funcreq{System powinien umożliwiać Administratorom zmianę uprawnień innych użytkowników.}{10}{4}
\funcreq{System powinien umożliwiać Administratorom usuwanie kont innych użytkowników.}{9}{4}
\funcreq{System powinien umożliwiać Administratorom robienie backupu danych.}{10}{7}
\funcreq{System powinien umożliwiać Administratorom przeglądanie logów systemu.}{10}{3}
\funcreq{System powinien umożliwiać Administratorom przeprowadzenie rekonfiguracji Systemu.}{10}{8}
\section{Moduł główny}
% Moduł główny
\funcreq{System powinien pozwolić użytkownikom na tworzenie nowego projektu.}{10}{5}
\funcreq{System powinien pozwolić Twórcom Projektu dodawanie Członków Projektu.}{10}{3}
\funcreq{System powinien pozwolić Członkom Projektu na przeglądanie Projektów do których należą.}{10}{2}
\funcreq{System powinien pozwolić Członkom projektu na tworzenie nowych zgłoszeń.}{10}{5}
\funcreq{System powinien pozwolić Członkom projektu przeglądać Zgłoszenia.}{10}{2}
\funcreq{System powinien pozwolić na przypisywanie osób odpowiedzialnych za zgłoszenie pośród Członków projektu}{9}{2}
\funcreq{System powinien pozwolić Osobom odpowiedzialnym za zgłoszenie zmianę jego statusu.}{10}{2}
\funcreq{System powinien pozwolić Twórcom zgłoszenie weryfikację jego realizacji i odpowiednią zmianę statusu.}{9}{2}
\funcreq{System powinien umożliwiać Twórcy projektu edycję swojego projektu.}{10}{4}
\funcreq{System powinien umożliwiać Twórcy projektu zgłoszenie swojego projektu do zamknięcia.}{10}{2}
\funcreq{System powinien umożliwiać Twórcy Zgłoszenia zamknięcie swojego Zgłoszenia.}{10}{2}
\funcreq{System powinien umożliwiać Twórcy Zgłoszenia zmianę widoczności swojego Zgłoszenia.}{5}{2}
\funcreq{System powinien umożliwiać Twórcy Zgłoszenia edycję podstawowych informacji o swoim Zgłoszeniu.}{9}{4}
\funcreq{System powinien umożliwiać Twórcy Zgłoszenia wyznaczenie Użytkownika będącego Osobą Odpowiedzialną za Zgłoszenie.}{9}{3}
\funcreq{System powinien umożliwiać Odbiorcy Zgłoszenia udzielenie informacji zwrotnej Osobie Odpowiedzialnej za Zgłoszenie.}{9}{3}
\funcreq{System powinien umożliwiać Twórcy Zgłoszenia planowanie kolejnych etapów Projektu w postaci grafu.}{10}{7}
\funcreq{System powinien umożliwiać Członkowi Projektu edycję Grafu poprzez dodawanie, edytowanie i usuwanie Zgłoszeń.}{10}{7}
\funcreq{System powinien umożliwiać Osobie Odpowiedzialnej aktualizowanie Zasobów Zgłoszenia.}{8}{5}
\funcreq{System powinien umożliwiać Osobie Odpowiedzialnej wysyłanie do Twórcy Zgłoszenia informacji, że Zgłoszenie jest do zamknięcia.}{6}{2}
\funcreq{System powinien umożliwiać Członkowi Projektu wysyłanie do Twórcy Projektu informacji, że Projekt jest do zamknięcia.}{6}{2}
\funcreq{System powinien umożliwiać Twórcy, Osobie Odpowiedzialnej, Odbiorcy pobieranie załączników dodanych do Zgłoszeń.}{4}{5}
\funcreq{System powinien umożliwiać Członkowi Projektu przeszukiwanie Zgłoszeń w Projekcie.}{8}{5}
\funcreq{System powinien umożliwiać Twórcy Projektu na zmianę uprawnień Członków Projektu w danym Projekcie}{5}{5}

\chapter{Wymagania niefunkcjonalne}
\section{Dostępność}
\nonfuncreq{System powinien być dostępny minimum 99\% czasu.}{10}{8}
\section{Bezpieczeństwo}
\nonfuncreq{System powinien robić backup bazy danych codziennie podczas małego obciążenia (prawdopodobnie o 3:15 w nocy). Nie powinno to powodować przerwy w dostępności.}{10}{6}
\nonfuncreq{Połączenie z serwerem jest szyfrowane.}{10}{6}
\nonfuncreq{System powinien generować logi zawierające informacje o zmianach dokonywanych przez użytkowników oraz informacje diagnostyczne.}{7}{4}
\nonfuncreq{System powinien wykrywać próby nieautoryzowanego dostępu i informować o nich administratora.}{7}{5}
\nonfuncreq{System powinien być blokowany na 10 min dla IP adresu z którego próbowano się połączyć więcej niż 5 razy.}{8}{6}
\section{Skalowalność}
\nonfuncreq{System powinien być skalowalny (dwukrotne zwiększenie ilości serwerów powinno pozwolić na dwukrotne zwiększenie ilości klientów/ kont użytkowników).}{7}{5}
\section{Ergonomia}
\nonfuncreq{System powinien być ergonomiczny, czyli wygodny w użyciu.}{10}{9}



\section{Kategoria do wyrzucenia}
\comment{Nie wiem do jakiej kategorii te wrzucić. I szczerze mówiąc nie do końca rozumiem o co w nich chodzi...} \todo
\nonfuncreq{Konstrukcja systemu powinna być taka, że w przypadku "zepsucia systemu" można w miarę szybko naprawić ten system.}{\todo}{\todo}
\nonfuncreq{System powinien być elastyczny, czyli ma mieć możliwość zmieniać swoje właściwości ze zmianą popytu na produkcję.}{\todo}{\todo}

\chapter{Diagramy sekwencji}
\section{Moduł uwierzytelniania}

\begin{sequence}{Logowanie}
\diagramSQ{logowanie_sekwencja.pdf}{1}
\end{sequence}

\begin{sequence}{Rejestracja}
\diagramSQ{rejestracja_sekwencja.pdf}{1}
\end{sequence}

\section{Moduł administracyjny}

\begin{sequence}{Zatwierdzanie konta}
\diagramSQ{zatwierdzenie_konta_sekwencja.pdf}{1}
\end{sequence}

\begin{sequence}{Usuwanie konta}
\diagramSQ{usuwanie_konta_sekwencja.pdf}{1}
\end{sequence}

\begin{sequence}{Zmiana uprawnień}
\diagramSQ{zmiana_uprawnien_sekwencja.pdf}{1}
\end{sequence}

\begin{sequence}{Backup danych}
\diagramSQ{backup_sekwencja.pdf}{1}
\end{sequence}

\begin{sequence}{Przywracanie systemu}
\diagramSQ{przywracanie_systemu_sekwencja.pdf}{1}
\end{sequence}

\begin{sequence}{Przeglądanie logów Systemu}
\diagramSQ{przegladanie_logow_sekwencja.pdf}{1}
\end{sequence}

\begin{sequence}{Rekonfiguracja Systemu}
\diagramSQ{rekonfiguracja_sekwencja.pdf}{1}
\end{sequence}

\section{Moduł główny}

\begin{sequence}{Przeglądanie Projektów}
	\label{Przegladanie projektow}
	\diagramSQ{przegladanie_projektow_sekwencja_fit.pdf}{0.9}
\end{sequence}

\begin{sequence}{Edycja Projektu}
	\label{Edycja projektu}
	\diagramSQ{edycja_projektu_sekwencja_fit.pdf}{0.9}
\end{sequence}

\begin{sequence}{Zamknięcie projektu}
	\label{Zamkniecie projektu}
	\diagramSQ{zamkniecie_projektu_sekwencja_fit.pdf}{0.9}
\end{sequence}

\begin{sequence}{Tworzenie zgłoszenia}
	\label{Tworzenie zgloszenia}
	\diagramSQ{tworzenie_zgloszenia_sekwencja_fit.pdf}{0.9}
\end{sequence}

\begin{sequence}{Zamknięcie zgłoszenia}
	\label{Zamkniecie zgloszenia}
	\diagramSQ{zamkniecie_zgloszenia_sekwencja_fit.pdf}{0.9}
\end{sequence}

\begin{sequence}{Zmaina widocznosci zgłoszenia}
	\label{Zmiana widocznosci zgloszenia}
	\diagramSQ{zmiana_widocznosci_zgloszenia_sekwencja_fit.pdf}{0.9}
\end{sequence}

\begin{sequence}{Edycja zgłoszenia}
	\label{Edycja zgloszenia}
	\diagramSQ{edycja_zgloszenia_sekwencja_fit.pdf}{0.9}
\end{sequence}




\chapter{Diagramy Klas}
\section{Ogólny diagram klas}

\diagramSQ{class_diagram_complete.pdf}{1.1}

\section{Moduł uwierzytelniania}

\begin{class}{Logowanie}
\diagramSQ{logowanie_klas.pdf}{1}
\end{class}

\begin{class}{Rejestracja}
	\diagramSQ{rejestracja_klas.pdf}{1}
\end{class}

\section{Moduł administracyjny}

\begin{class}{Zatwierdzanie konta}
	\diagramSQ{zatwierdzenie_konta_klas.pdf}{1}
\end{class}

\begin{class}{Usuwanie konta}
	\diagramSQ{usuwanie_konta_zmiana_uprawnien_klas.pdf}{1}
\end{class}

\begin{class}{Backup}
	\diagramSQ{backup_klas.pdf}{1}
\end{class}

\begin{class}{Przywracanie Systemu}
	\diagramSQ{przywracanie_systemu_klas.pdf}{1}
\end{class}

\begin{class}{Przeglądanie Logów}
	\diagramSQ{przegladanie_logow_klas.pdf}{1}
\end{class}

\begin{class}{Zmiana uprawnień}
	\diagramSQ{usuwanie_konta_zmiana_uprawnien_klas.pdf}{1}
\end{class}

\begin{class}{Rekonfiguracja Systemu}
	\diagramSQ{rekonfiguracja_klas.pdf}{1}
\end{class}

\section{Moduł główny}
\begin{class}{Edycja, przeglądanie i zamknięcie projektu}
	\diagramSQ{projekt_actions.pdf}{1}
\end{class}

\begin{class}{Edycja, przeglądanie, zmiana widocznosci i zamknięcie zgloszenia}
	\diagramSQ{task_actions.pdf}{1}
\end{class}





\chapter{Diagram komponentów}
%jednak bez tego bo bedzie ogolny jeden, wiec nie trzeba liczyc
%\begin{component}{Diagram komponentów całego systemu}
\diagramC{diagram_komponentow.png}{1}
%\end{component}



\chapter{Diagram bazy danych}
%jednak bez tego bo bedzie ogolny jeden, wiec nie trzeba liczyc
%\begin{component}{Diagram komponentów całego systemu}
\diagramD{database_grapics.pdf}{1}
%\end{component}



\chapter{Prototyp interfejsu}
W tym rozdziale przedstawione zostały prototypowe wersje wybranych elementów interfejsu użytkownika.
\newcommand{\uiscale}{0.6}
\begin{figure}[ht!]
	\centering
	\includegraphics[scale=\uiscale]{ui/login.png}
	\caption{Okno logowania}
\end{figure}
\begin{figure}[ht!]
	\centering
	\includegraphics[scale=\uiscale]{ui/register.png}
	\caption{Okno rejestracji}
\end{figure}
\begin{figure}[ht!]
	\centering
	\includegraphics[scale=\uiscale]{ui/projects.png}
	\caption{Okno przeglądu projektów}
\end{figure}
\begin{figure}[ht!]
	\centering
	\includegraphics[scale=\uiscale]{ui/issues.png}
	\caption{Okno przeglądu zgłoszeń w projekcie}
\end{figure}
\begin{figure}[ht!]
	\centering
	\includegraphics[scale=\uiscale]{ui/edit_issue.png}
	\caption{Okno edycji zgłoszenia}
\end{figure}

\end{document}
